\documentclass{article}

% Encoding and lang
\usepackage[T1]{fontenc}
\usepackage[utf8]{inputenc}
\usepackage[english]{babel}

% Graphical packages
\usepackage{graphicx}
\usepackage{xcolor}
\usepackage{xspace}

% Math
\usepackage{amsmath}
\usepackage{amsfonts}
\usepackage{amssymb}
\usepackage{amsthm}
\usepackage{thm-restate}
% \usepackage{mathrsfs}
\usepackage{mathtools}
\usepackage{textcomp}
\usepackage{gensymb}
% \usepackage{textgreek}
\usepackage{multicol}
\usepackage{ stmaryrd }

\usepackage{fullpage}

\theoremstyle{definition}
\newtheorem{theorem}{Theorem}
\newtheorem{lemma}{Lemma}
\newtheorem{corollary}{Corollary}
\newtheorem{definition}{Definition}

\usepackage{minted}
\newmintinline[ocaml]{ocaml}{}
\usepackage[scaled=0.85]{DejaVuSansMono}

\usepackage{caption}
\usepackage{subcaption}
\usepackage[inline,shortlabels]{enumitem}
\setlist{leftmargin=*,noitemsep}
\usepackage{array}
\usepackage{bm}%Decent bolding for math symbols

\usepackage{natbib}% Good citations and bibliography
\usepackage{mathpartir} % Syntax trees

\usepackage[pdfusetitle,colorlinks=true,citecolor=Rhodamine]{hyperref}
\usepackage[noabbrev,capitalize,nameinlink]{cleveref}
\SetLabelAlign{parright}{\parbox[t]{\labelwidth}{\raggedleft#1}}

\newcommand\TODO[1]{{\textbf{\color{red}{TODO: #1}}}}

\title{A parsimonious module system}
\author{TODO}
\date{}

\newcommand\htag[1]{\shortintertext{\textbf{#1}}}
\newcommand\ddotop{\operatorname{:}}
\newcommand\Multi[2][{}]{\overline{#2}}
\newcommand\rg[2]{\left[#1;#2\right]}

%% Term
\newcommand\im{M}
\newcommand\iv{V}
\newcommand\is{S}
\newcommand\id{D}
\newcommand\ip{P}
\newcommand\ix{X}

%% Types
\newcommand\iM{\mathcal M}
\newcommand\iMO{{\mathcal M^\circ}}
\newcommand\iS{\mathcal S}
\newcommand\iD{\mathcal D}
\newcommand\iX{\mathcal X}
\newcommand\iC{\mathcal C}
\newcommand\iP{\mathcal P}

%% Syntax
\newcommand\Sapp[2]{#1(#2)}
\newcommand\Sconstraint[2]{(#1 \ddotop #2)}
\newcommand\Stransp[2]{(#1 \operatorname{<:} #2)}
\newcommand\Sfunctor[3]{\Sconstraint{#1}{#2} \to #3}
\newcommand\Salias[1]{(= #1)}
\newcommand\Swith[2]{#1\ \mathtt{with}\ #2}

\newcommand\Sstruct[1]{\mathtt{struct}\ #1\ \mathtt{end}}
\newcommand\Ssig[1]{\mathtt{sig}\ #1\ \mathtt{end}}
\newcommand\Sempty\varepsilon

\newcommand\Senrich[2]{{#2}_{\color{black}\left[#1\right]}}
\newcommand\Sstrengthen[3]{#3/_{#1}#2}

\newcommand\Slet[2]{\mathtt{let}\ #1 = #2}
\newcommand\Sval[2]{\mathtt{val}\ #1 : #2}
\newcommand\Stype[2]{\mathtt{type}\ #1 = #2}
\newcommand\Stypeabs[1]{\mathtt{type}\ #1}
\newcommand\Smodule[2]{\mathtt{module}\ #1 = #2}
\newcommand\Smodulety[2]{\mathtt{module}\ #1 : #2}
\newcommand\Smoduletype[2]{\mathtt{module\ type}\ #1 = #2}

%% Env
\newcommand\E{\Gamma}
\newcommand\lookup[2]{#1(#2)}
\newcommand\resolve[2]{\operatorname{resolve}_{#1}(#2)}

% Operation
\newcommand\Oenrich[2]{{#2}_{\color{red}\left[#1\right]}}
\newcommand\Ostrengthen[3]{#3{\color{red}/_{#1}#2}}
\newcommand\Otransp[2]{(#1 \operatorname{{\color{red}<:}} #2)}
\newcommand\Oforce[2]{\operatorname{force}_{#1}(#2)}
\newcommand\Onormalize[2]{\operatorname{normalize}_{#1}(#2)}
\newcommand\Osubst[2]{\mathtt{let}\ #1\ \mathtt{in}\ #2}

\newcommand\subst[3]{#3\!\left[#1 \mapsto #2\right]}
\newcommand\substs[4]{#4\!\left[#1 \mapsto #2\ \middle|\ #1\in#3\right]}
\newcommand\bv[1]{\operatorname{BV}(#1)}

%% Jugements

\newcommand\iswt{\operatorname{\triangleright}}
\newcommand\iswtm{\operatorname{\blacktriangleright}}
\newcommand\wt[4][]{#2 \iswt_{#1} #3 \ddotop #4}
\newcommand\wtp[3]{#1 \iswtm #2 \ddotop #3}
\newcommand\wtm[4][]{#2 \iswtm_{#1} #3 \ddotop #4}

% Welformedness of types and terms
\newcommand\iswf{\operatorname{\vDash}}
\newcommand\wf[3][]{#2 \iswf_{#1} #3}
\newcommand\wfp[2]{#1 \iswf #2}
\newcommand\wfm[3][]{#2 \iswf_{#1} #3}

% Module inclusion
\newcommand\issub{<:}
\newcommand\subtyp[4][]{#2 \iswt_{#1} #3 \issub #4}
\newcommand\submod[5][]{#2 \iswtm_{#1} #3 \issub #4 \leadsto #5}
\newcommand\subpath[4][]{#2 \iswtm_{#1} #3 \lesssim #4}


%%% Local Variables:
%%% mode: latex
%%% TeX-master: "main.tex"
%%% End:


\begin{document}

\maketitle

\section{Introduction}

We propose an \emph{expressive}, \emph{syntactic} and \emph{parsimonious} module system.
By expressive, we mean that we support a rich subset of features that have so far
not been formalized together and allow the user for rich manipulation of modules, notably transparent ascriptions, module aliases, enrichment constraints and
applicative functors. Unlike work like \cite{fing}, our rules are
defined \emph{syntactically}, directly on the concrete syntax. This makes the rule easier to grasp, make deriving an inference algorithm almost immediate, and
ensure that error messages and typing information are easier to surface to the user.
Finally, this system is \emph{parsimonious}, in that it does the minimum amount of
expansion required by using ideas similar to explicit substitutions for
module operations. Module types can be quite large, and limiting the expansion
can simplify error messages and improve the speed of the typechecker.

TODO: Explain functor aliasing.

%%% Local Variables:
%%% mode: latex
%%% TeX-master: t
%%% End:

\section{Motivations}
\label{motivations}

\subsection{The aliasing problem}

Several advanced module systems allow ``module aliases''. If we consider a
declaration of the form:

\begin{minted}{ocaml}
module Alias = Original
\end{minted}

In the usual presentation, {\tt Alias} would have the module type {\tt S},
which is the expanded module type of {\tt Original}.
With module aliases, {\tt Alias} is exactly of type \ocaml{(= Original)},
i.e., a singleton type \cite{XX} that uniquely
identify it as an alias of {\tt Original}. This works well with the applicative
behavior of functor used notably by OCaml,
allowing us to preserve \ocaml{F(Alias) = F(Original)}
for any applicative functor {\tt F}.

Unfortunately, this also leads to subtle issues with functors and subtyping.
Let us consider the following examples:

\begin{minted}{ocaml}
module F (X : S) = struct
  module A = X (* this is not an alias *)
end
\end{minted}

What should be the type of \ocaml/F(M).A/? If we keep the singleton type
\ocaml/A : (= X)/, we would have \ocaml/F(M).A : (= M)/. This would mean that
\ocaml{F(M).A} is exactly the same module as \ocaml{M}. This
is however not the case: indeed, when given as argument to the functor,
\ocaml{M} was restricted, by subtyping, to the signature \ocaml{S}.
Inside the body of the functor, we only get a restricted view of {\tt M} through
the lense of {\tt S} and {\tt A} can only present this restricted view.
In practice, this subtyping is implemented by copy: {\tt X} is a new module
which only contains the fields of {\tt M} that are visible in {\tt S}.
If users outside the functor can observe the equality \ocaml/F(M).A = M/,
they might try to access to fields that were not copied and are thus not
really present in the module.
\TODO{Make all this a bit clearer, probably with a schema}

One solution to this problem is \emph{transparent ascriptions}, a type of ascription
that still hides values, but purposefully ``leaks through'' types and other
static fields.
We propose to add transparent ascriptions directly in module paths, allowing
to use them in module aliases. This gives us an immediate
solution to the identified unsafety while still preserving all the possible sharing.
We obtain the following type for the example above:

\begin{minted}{ocaml}
module F (X : S) : sig
  module A = X (* This is a real alias *)
end
\end{minted}
However, once we apply \mintinline{ocaml}{F}, we obtain:

\begin{minted}{ocaml}
module Res = F(M)
> Module Res : sig
>   module A = (M <: S) (* X viewed through S *)
> end
\end{minted}

Since we now have \ocaml/F(M).A = (M <: S)/, the sharing
between {\tt A} and {\tt M} is properly exposed,
but it still restricts the access to dynamic fields that are not
present in {\tt S}.

\subsection{The avoidance problem}

TODO: Explain avoidance problem more

\begin{minted}{ocaml}
module F (X : S) = struct
  type a = X.t list
  type b = X.t * int
end
\end{minted}

What's the type of:
\begin{minted}{ocaml}
module A = F(struct type t = A | B end)
\end{minted}

The typechecker gives us the following un-helpful answer
\begin{minted}{ocaml}
module A : sig type a type b end
\end{minted}

We propose to use local modules:

\begin{minted}{ocaml}
module A : let X : ... in sig
  type a = X.t list
  type b = X.t * int
end
\end{minted}

\subsection{Incrementality in module checking}

Modern practical module systems \cite{Rich calculus, etc} provide
many additional operators not present in the original formulation.
For instance in OCaml, ``with constraints'', noted \ocaml{S with type t = int}
allow to add an additional equality to a module type \ocaml{S}.
\TODO{Cite other operations}.

This operation are so far only syntactic: they are expanded immediately into
a complete signature. For instance 
\begin{minted}{ocaml}
module type S = sig 
  val x : string
  module A : T
end
module M : S with type A.t = int
\end{minted}

is immediately expanded as:
\begin{minted}{ocaml}
module M : sig  val x : string  module A : sig type t = int ... end  end
\end{minted}

This yield a simple semantics, but causes numerous efficiency issues in
practical systems that support large Ecosystems: indeed, the signature
of the expanded version can be much larger than the original one.
Our calculus, on the contrary, only expand when absolutely necessary, and keep
the simplest form otherwise. This gives a presentation
similar to explicit substitutions in more traditional type theories, with
similar problematic and trade-offs: we want to minimize the amount of work,
while still doing enough to ensure we do not delay errors.

The application of this technique throughout our system lead to a novel
presentation of parsimonious subtyping that only pushes ascriptions up
to aliases, but not further, ensuring that we maximize the sharing described
in the previous section.


\subsection{Contributions}

We propose a novel module systems which:
\begin{itemize}
\item Uses \emph{syntactic} rules, making it easier to reason about and
  implement as part of production typechecker with inference and error reporting
\item Support \emph{rich operations} on modules such as strengthening (\cite{XL}),
  with constraints, \dots  
\item Support \emph{transparent ascriptions in path}, which safely and
  conveniently extend
  the applicative behavior of functors to a context with singletons.
\item Computes types and subtyping \emph{parsimoniously},
  using an approach based on explicit substitutions and input/output judgements.
\end{itemize}



%%% Local Variables:
%%% mode: latex
%%% TeX-command-extra-options: "-shell-escape"
%%% TeX-master: "main.tex"
%%% End:

\section{An ML module calculus}

We now introduce our module calculus.
First, let us defined some conventions:
lowercase meta-variables such as $x$, $e$ or $t$ are used for the core language;
capitalized meta-variables such as $\ix$, $\im$, $\is$ represent modules;
calligraphic meta-variables such as $\iX$, $\iM$, $\iS$ represent module types.

\subsection{Syntax}

The syntax is defined in \cref{grammar}.
The module expression language contains paths $\ip$, which might
be variables $\ix$ or qualified accesses $\ip.\ix$,
parameterized modules i.e.~\emph{functors} ($\Sfunctor{\ix}{\iM}{\im}$),
module type annotations i.e.~\emph{ascriptions}, both
opaque $\Sconstraint{\im}{\iM}$ and transparent $\Stransp{\im}{\iM}$,
and finally structures $\Sstruct{\ix}{\is}$. Structures
contains a list of declarations, noted $\id$, which can be value, type, module or module type bindings. Structures are also annotated with a ``Self'' variable,
which represent the object through which
other field in the module should be accessed.

In our core calculus, module types are stratified. $\iM$ represent module
types with operations, while $\iMO$ are raw module types without initial operations
(i.e., in head normal form). Raw module types can be (qualified)
module type variables $\iX$ or $\iP.\iX$, functor types
$\Sfunctor{\iX}{\iM_1}{\iM_2}$, signatures $\Ssig{\ix}{\iS}$,
and singletons $\Salias{\iP}$.
As with structures, signatures are a list of declarations noted $\iD$, and
a ``Self'' variable.
Allowed operations strengthening a module by a path,
noted $\Sstrengthen{\iP}{\iM}$,
enriching a module with additional constraints, noted $\Senrich{\iC}{\iM}$,
and locally binding a module name, noted $\Ssubst{\ix : \iM}{\iM'}$.

As described before, there are two kinds of paths.
The first, $\ip$, is
mostly used for \emph{dynamic} components, and is composed of a list of
modules of the form $A.B.C$.
The second, $\iP$, is mostly used for \emph{static} components such as types,
and can additionally contain functor applications and transparent ascriptions,
for instance $F(\ix).\iy$ or $\Stransp{\ix}{\iM}.\iy$

Finally, the core language is left largely undefined in our calculus, except for qualified accesses.

$\subst{\ix}{\iy}{\iM}$ denotes the substitution of variable $\ix$ by variable $\iy$ in $\iM$. We only substitute variables by other variables.

\begin{figure}[!hb]
  
\begin{subfigure}[t]{0.45\linewidth}
\begin{align*}
  \htag{Path}
  \ip ::=&\ \ix \mid \ip.\ix\\
  \iP ::=&\ \ix \mid \iP.\ix \mid \Sapp{\iP_1}{\iP_2} \mid \Stransp{\iP}{\iM}
  \htag{Module Expressions}
  \im ::=&\ \ip\tag{Variables}\\
  |&\ \Sconstraint{\im}{\iM}\tag{Opaque Ascription}\\
  |&\ \Stransp{\im}{\iM}\tag{Transparent Ascription}\\
  |&\ \Sapp{\im_1}{\im_2}\tag{Functor application}\\
  |&\ \Sfunctor{\ix}{\iM}{\im}\tag{Functor}\\
  |&\ \Sstruct{\ix}{\is}\tag{Structure}\\
  \htag{Structures}
  \is ::=&\ \Sempty\ |\ \id;\is\\
  \id ::=&\ \Slet{x}{e}\tag{Values}\\
  |&\ \Stype{t}{\tau}\tag{Types}\\
  |&\ \Smodule{\ix}{\im}\tag{Modules}\\
  |&\ \Smoduletype{\iX}{\iM}\tag{Module types}\\
  \htag{Core language}
  e ::=&\ \ip.x \tag{Qualified variable}\\
  |&\ \dots \tag{Other expressions}\\
  \tau ::=&\ {\iP.t} \tag{Qualified type}\\
  |&\ \dots \tag{Other types}
\end{align*}
\end{subfigure}\hfill
\begin{subfigure}[t]{0.5\linewidth}
\begin{align*}
  \htag{Module types}
  \iMO ::=&\ \iX\ |\ \iP.\iX\tag{Variables}\\
  |&\ \Salias{\iP}\tag{Alias}\\
  |&\ \Sfunctor{\iX}{\iM_1}{\iM_2}\tag{Functor}\\
  |&\ \Ssig{\ix}{\iS}\tag{Signature}\\
  \iM ::=&\ \Sstrengthen{l}{\iP}{\iM}\tag{Strengthening}\\
  |&\ \Ssubst{\ix : \iM}{\iM}\tag{Let}\\
  |&\ \Senrich{\iC}{\iM}\tag{Enrichment}\\
  |&\ \iMO\\
  \htag{Enrichment}
  \iC ::=&\ \ip.t = \tau\\
  |&\ \ip : \iM\\
  \htag{Signatures}
  \iS ::=&\ \Sempty\ |\ \iD;\iS\\
  \iD ::=&\ \Sval{x}{\tau}\tag{Values}\\
  |&\ \Stype{t}{\tau}\tag{Types}\\
  |&\ \Stypeabs{t}\tag{Abstract types}\\
  |&\ \Smodulety{\ix}{\iM}\tag{Modules}\\
  |&\ \Smoduletype{\iX}{\iM}\tag{Module types}\\
  \htag{Environments}
    \E ::=&\ \iS
\end{align*}
\end{subfigure}

%%% Local Variables:
%%% mode: latex
%%% TeX-master: "../main"
%%% End:

  \caption{Module language}
  \label{grammar}
\end{figure}


\subsection{Notations}

Let us now define a few notations that we will use in the rest of this
article. We will then describe how these various construction interact
on specific features.

We define the following type judgements on modules. Note that the subtyping
judgement also \emph{returns} a module type, which is the result
of the transparent ascription between the two modules.
\begin{description}[align=right, leftmargin=3.5cm]
\item[$\wtm{\E}{\im}{\iM}$ :]
  The module $\im$ is of type $\iM$ in $\E$.
  See \cref{module:typing}.
\item[$\submod{\E}{\iM}{\iM'}{\iM_r}$ :]
  The module type $\iM$ is a subtype of $\iM'$ in environment $\E$,
  and their ascription is $\iM_r$.
  See \cref{module:subtyping}.
\item[$\Oreduce{\E}{\iM} = \iMO$ :]
  The module type with operations $\iM$ can be reduced to a module
  type without operations $\iMO$ in environment $\E$.
  See \cref{module:strengthen,module:enrich,module:subst}.
  \TODO{Change bracket style ...}
\end{description}

In our calculus, environments are simply signature on which we can
``query'' fields.
To properly model our module calculus, we need to carefully consider
how these environments are accessed. For this purpose, we consider the following
operations on environments, which are defined in \cref{module:envaccess}
\begin{description}[align=right, leftmargin=3.5cm]
\item[$\lookup{\E}{\ip}$ :]
  Lookup the path $\ip$ in $\E$ to a module or module type.
\item[$\Onormalize{\E}{\iP}$ :]
  Normalizes the path $\iP$ in $\E$ to a canonical path.
\item[$\resolve{\E}{\iP}$ :]
  Resolve the path $\iP$ in $\E$ to its shape (i.e., either an arrow or a signature).
\item[$\Oshape{\E}{\iM}$ :]
  Resolve the module $\iM$ in $\E$ to its shape (i.e., either an arrow or a signature).
\end{description}

\subsection{Structures and qualified accesses}

\subsection{Subtyping and transparent ascriptions}

\subsection{Functors}

\subsection{Module operations}

\begin{figure}[tbp]
  \mprset{sep=1.5em}
  \begin{mathpar}
  % \inferrule[ModVar]
  % { \Smodulety{\ix}{\iM} \in \E }
  % { \wtm{\E}{\ix}{\Salias{\ix}} }
  % \and
  \inferrule[ModVar]
  { \ip \in \E }
  % { \wtm{\E}{\ip.\ix}{\substs{n}{\ip.n}{\bv{\iS_1}}{\iM}} }
  { \wtm{\E}{\ip}{\Salias{\ip}} }
  \and
  % \inferrule[Strength]
  % { \wtm{\E}{\ip}{\iM} }
  % { \wtm{\E}{\ip}{\Sstrengthen{}{\ip}{\iM}} }
  % \and
  % \inferrule
  % { \wtm{\E}{\im}{\iM'} \\ \submod{\E}{\iM'}{\iM}{\_} }
  % { \wtm{\E}{\im}{\iM} }
  % \and
  \inferrule
  { \wtm{\E}{\im}{\iM'} \\\submod{\E}{\iM'}{\iM}{\_} }
  { \wtm{\E}{\Sconstraint{\im}{\iM}}{\iM} }
  \and
  \inferrule
  { \wtm{\E}{\im}{\iM'} \\ \submod{\E}{\iM'}{\iM}{\iM_r} }
  { \wtm{\E}{\Stransp{\im}{\iM}}{\iM_r} }
  %
  \and
  \inferrule
  { \wtm{\E}{\im_f}{\Sfunctor{\ix}{\iM_a}{\iM_r}} \\
    \iy \text{ fresh}\\
    \wtm{\E}{\im}{\iM} \\
    \submod{\E}{\iM}{\iM_a}{\iM_c}
  }
  { \wtm{\E}{\Sapp{\im_f}{\im}}
    {\Ssubst{\iy : \iM_c}{\subst{\ix}{\iy}{\iM_r}}} }
  \and
  % \inferrule
  % { \wtm{\E}{\im_f}{\Sfunctor{\ix}{\iM}{\iM_r}} \\
  %   \wtm{\E}{\im_a}{\iM_a} \\
  %   \submod{\E}{\iM_a}{\iM}
  % }
  % { \wtm{\E}{\Sapp{\im_f}{\im_a}}{\iM_r} }
  % \and
  \inferrule
  { \wfm{\E}{\iM} \\
    X \notin \E \\
    \wtm{\E;\Smodulety{X}{\iM}}{\im}{\iM'}
  }
  { \wtm{\E}{\Sfunctor{X}{\iM}{\im}}{\Sfunctor{X}{\iM}{\iM'}} }
  \and
  \inferrule
  { \wtm[\ix]{\E}{\is}{\iS} }
  { \wtm{\E}{\Sstruct{\ix}{\is}}{\Ssig{\ix}{\iS}} }
  \and
  \inferrule
  { }
  { \wtm[\iy]{\E}{\Sempty}{\Sempty} }
  %
  \\
  \inferrule
  { \wt{\E}{e}{\tau} \\
    x \notin \lookup{\E}{\iy} \\
    \wtm[\iy]{\E;\Sval{\iy.x}{\tau}}{\is}{\iS}
  }
  { \wtm[\iy]{\E}{(\Slet{x}{e}; \is)}{(\Sval{x}{\tau}; \iS)} }
  \and
  \inferrule
  { \wtm{\E}{\im}{\iM} \\
    % \submod{\E}{\Sstrengthen{}{\im}{\iM'}}{\iM} \\
    \ix \notin \lookup{\E}{\iy} \\
    \wtm[\iy]{\E;\Smodulety{\iy.\ix}{\iM}}{\is}{\iS}
  }
  { \wtm[\iy]{\E}{(\Smodule{\ix}{\im}; \is)}{(\Smodulety{\ix}{\iM}; \iS)} }
  \and
  \inferrule
  { \wf{\E}{\tau} \\
    t \notin \lookup{\E}{\iy} \\
    \wtm[\iy]{\E;\Stype{\iy.t}{\tau}}{\is}{\iS}
  }
  { \wtm[\iy]{\E}{(\Stype{t}{\tau}; \is)}{(\Stype{t}{\tau}; \iS)} }
  \and
  \inferrule
  { \wfm{\E}{\iM} \\
    \iX \notin \lookup{\E}{\iy} \\
    \wtm[\iy]{\E;\Smoduletype{\iy.\iX}{\iM}}{\is}{\iS}
  }
  { \wtm[\iy]{\E}{(\Smoduletype{\iX}{\iM}; \is)}{(\Smoduletype{\iX}{\iM}; \iS)} }
  % \and
  % \inferrule
  % { \wtm[\sideX]{\E}{(\struct{\is})}{(\signature{\valm[
  % { \wtp{\E}{(\prog{\is})}{\tau} }
\end{mathpar}

%%% Local Variables:
%%% mode: latex
%%% TeX-master: "../main"
%%% End:
\vspace{-3mm}
  \caption{Module typing rules -- $\wtm{\E}{\im}{\iM}$}
  \label{module:typing}
\end{figure}

\begin{figure}[tbp]
  \begin{mathpar}
  \inferrule
  { \submod{\E}{\Oforce{\E}{\iM}}{\Oforce{\E}{\iM'}}{\iM''} }
  { \submod{\E}{\iM}{\iM'}{\iM''} }
  \and
  \inferrule
  { \Onormalize{\E}{\iP} = \Onormalize{\E}{\iP'} }
  { \submod{\E}{\Salias{\iP}}{\Salias{\iP'}}{\Salias{\iP'}} }
  \and
  % \inferrule
  % { }
  % { \submod{\E}{\ip.\iX}{\ip.\iX} }
  % \and
  % \inferrule
  % { \submod{\E}{\E(\ip)}{\iMO} }
  % { \submod{\E}{\Salias{\ip}}{\iMO} }
  % \and
  % \inferrule
  % { \submod{\E}{\E(\ip.\iX)}{\iMO} }
  % { \submod{\E}{\ip.\iX}{\iMO} }
  % \and
  % \inferrule
  % { \submod{\E}{\iMO}{\E(\ip.\iX)} }
  % { \submod{\E}{\iMO}{\ip.\iX} }
  % \and
  \inferrule
  { \submod{\E}{\resolve{\E}{\iP}}{\iMO}{\iM} }
  { \submod{\E}{\Salias{\iP}}{\iMO}{\iM} }
  \and
  \inferrule
  { \submod{\E}{\lookup{\E}{\iP}}{\iMO}{\iM} }
  { \submod{\E}{\iP}{\iMO}{\iM} }
  \and
  \inferrule
  { \submod{\E}{\iMO}{\lookup{\E}{\iP}}{\iM} }
  { \submod{\E}{\iMO}{\iP}{\iM} }
  \and
  \inferrule
  { \Onormalize{\E}{\iP} = \Onormalize{\E}{\iP'} \\
    \submod{\E}{\resolve{\E}{\iP}}{\iM}{\iM''} \\
    \submod{\E}{\iM''}{\iM'}{\_}
  }
  { \submod{\E}
    {\Salias{\Stransp{\iP}{\iM}}}{\Salias{\Stransp{\iP'}{\iM'}}}
    {\Salias{\Stransp{\iP'}{\iM'}}} }
  \and
  \inferrule
  { \subpath{\E}{\iP}{\iP'} \\
    \submod{\E}{\resolve{\E}{\iP}}{\iM}{\iM''} \\
    \submod{\E}{\iM''}{\resolve{\E}{\iP}}{\_}
  }
  { \submod{\E}{\Salias{\Stransp{\iP}{\iM}}}{\Salias{\iP'}}
    {\Salias{\iP'}} }
  \and
  % \inferrule
  % { \submod{\E}{\Multi\iC}{\Multi\iC'} }
  % { \submod{\E}{\Swith{\iM}{\Multi\iC}}{\Swith{\iM}{\Multi\iC'}} }
  % \and
  \inferrule
  { \submod{\E}{\iM'_a}{\iM_a}{\_} \\
    \submod{\E,\Smodulety{X}{\iM'_a}}{\iM_r}{\iM'_r}{\iM''_r}
  }
  { \submod{\E}{\Sfunctor{X}{\iM_a}{\iM_r}}{\Sfunctor{X}{\iM'_a}{\iM'_r}}
    {\Sfunctor{X}{\iM'_a}{\iM''_r}} }
  \and
  %
  \inferrule[\TODO{figure out permutation}]
  { \pi:\rg{1}{m} \to \rg{1}{n} \\
    \forall i\in[1;m],\ \submod{\E;\iD_1;\dots;\iD_n}{\iD_{\pi(i)}}{\iD'_i}{\iD''_i}
  }
  { \submod{\E}{\Ssig{\iD_1;\dots;\iD_n}}{\Ssig{\iD'_1;\dots;\iD'_m}}{\Ssig{\iD''_1;\dots;\iD''_m}} }
  \and
  %
  \inferrule
  { \submod{\E}{\iM_1}{\iM_2}{\iM}  }
  { \submod{\E}
    {(\Smodulety{X_i}{\iM_1})}{(\Smodulety{X_i}{\iM_2})}
    {(\Smodulety{X_i}{\iM})} }
  \and
  \inferrule
  { }
  { \submod{\E}
    {(\Stypeabs{t_i})}
    {(\Stypeabs{t_i})}
    {(\Stypeabs{t_i})}
  }
  \and
  \inferrule[\TODO{Double check which side to keep}]
  { \subtyp{\E}{\tau_1}{\tau_2}{} }
  { \submod{\E}
    {(\Sval{x_i}{\tau_1})}{(\Sval{x_i}{\tau_2})}
    {(\Sval{x_i}{\tau_2})} }
  \and
  \inferrule[\TODO{Double check which side to keep}]
  { \subtyp{\E}{\tau_1}{\tau_2} }
  { \submod{\E}
    {(\Stype{t_i}{\tau_1})}
    {(\Stype{t_i}{\tau_2})}
    {(\Stype{t_i}{\tau_1})}
  }
  \and
  \inferrule
  { \subtyp{\E}{t}{\tau} }
  { \submod{\E}
    {(\Stypeabs{t_i})}
    {(\Stype{t_i}{\tau})}
    {(\Stype{t_i}{\tau})}
  }
  \and
  \inferrule
  { }
  { \submod{\E}
    {(\Stype{t_i}{\tau})}
    {(\Stypeabs{t_i})}
    {(\Stype{t_i}{\tau})}
  }
\end{mathpar}

%%% Local Variables:
%%% mode: latex
%%% TeX-master: "../main"
%%% End:
\vspace{-3mm}
  \caption{Module subtyping rules -- $\submod{\E}{\iM}{\iM'}{\iM_r}$}
  \label{module:subtyping}
\end{figure}

\begin{figure}[tbp]
  \begin{align*}
  \Ostrengthen{\E}{\iP}{\iX}
  &\to \Ostrengthen{\E}{\iP}{\lookup{\E}{\iX}}\\
  \Ostrengthen{\E}{\iP}{\iP'.\iX}
  &\to \Ostrengthen{\E}{\iP}{\lookup{\E}{\iP'.\iX}}\\
  \Ostrengthen{\E}{\iP}{\Salias{\iP'}}
  &\to \Salias{\iP'}\\
  % \Ostrengthen{\E}{\iP}{(\Swith{\iM}{\iC})}
  % &\to \Swith{(\Ostrengthen{\E}{l,\operatorname{id}(\iC)}{\iP}{\iM})}{\iC}\\
  % &\to \Ostrengthen{\E}{\iP}{\Oenrich{\iC}{\iM}} \\
  \Ostrengthen{\E}{\iP}{(\Sfunctor{\ix}{\iM}{\iM'})}
  &\to \Sfunctor{\ix}{\iM}{\Sstrengthen{\iP(X)}{\iM'}}\\
  % \Ostrengthen{\E}{\iP}{\Stransp{\iM}{\iM'}}
  % &\to \Stransp{\Ostrengthen{\E}{\iP}{\iM}}{\iM'}\\
  \Ostrengthen{\E}{\iP}{\Ssig{\ix}{\iS}}
  &\to \Ssig{\ix}{\Ostrengthen{\E}{\iP}{\iS}}\\[4mm]
  %
  \Ostrengthen{\E}{\iP}{\Stype{t}{\tau};\iS}
  &\to \Stype{t}{\iP.t};\Ostrengthen{\E}{\iP}{\iS} % &\text{when } t\notin l
  \\
  % \Ostrengthen{\E}{\iP}{\Stype{t}{\tau};\iS}
  % &\to \Stype{t}{\tau};(\Ostrengthen{\E}{\iP}{\iS}) &\text{when } t\in l\\
  \Ostrengthen{\E}{\iP}{\Stypeabs{t};\iS}
  &\to \Stype{t}{\iP.t};\Ostrengthen{\E}{\iP}{\iS} % &\text{when } t\notin l
  \\
  % \Ostrengthen{\E}{\iP}{\Stypeabs{t};\iS}
  % &\to \Stypeabs{t};(\Ostrengthen{\E}{\iP}{\iS}) &\text{when } t\in l\\
  % \Ostrengthen{\E}{\iP}{\Smodulety{\ix}{\iM};\iS}
  % &\to \Smodulety{\ix}{\Ostrengthen{\E}{\operatorname{chop}(l,\ix)}{\iP.\ix}{\iM}};
  %   (\Ostrengthen{\E}{\iP}{\iS}) &\text{when } \ix\notin l\\
  \Ostrengthen{\E}{\iP}{\Smodulety{\ix}{\iM};\iS}
  &\to \Smodulety{\ix}{\Salias{\iP.\ix}};
    \Ostrengthen{\E}{\iP}{\iS} % &\text{when } \ix\notin l
  \\
  % \Ostrengthen{\E}{\iP}{\Smodulety{\ix}{\iM};\iS}
  % &\to \Smodulety{\ix}{\iM};
  %   (\Ostrengthen{\E}{\iP}{\iS})  &\text{when } \ix\in l\\
  \Ostrengthen{\E}{\iP}{\Smoduletype{\iX}{\iM};\iS}
  &\to \Smoduletype{\iX}{\iM};
    \Ostrengthen{\E}{\iP}{\iS} \\[4mm]
  %
  \Ostrengthen{\E}{\iP'}{\Sstrengthen{\iP}{\iM}}
  &\to \Ostrengthen{\E}{\iP}{\iM}\\
  \Ostrengthen{\E}{\iP}{\Senrich{\iC}{\iM}}
  &\to \Ostrengthen{\E}{\iP}{\Oenrich{\E}{\iC}{\iM}}\\
\end{align*}\vspace{-3mm}

%%% Local Variables:
%%% mode: latex
%%% TeX-master: "../main.tex"
%%% End:

  \caption{Module strengthening operation -- $\Sstrengthen{\iP}{\iM}$}
  \label{module:strengthen}
\end{figure}

\begin{figure}[tbp]
  \begin{align*}
  \Oenrich{\E}{\iC}{\iX}
  &\to \Oenrich{\E}{\iC}{\lookup{\E}{\iX}}\\
  \Oenrich{\E}{\iC}{\iP.\iX}
  &\to \Oenrich{\E}{\iC}{\lookup{\E}{\iP.\iX}}\\
  % \Oenrich{\E}{\iC}{(\Swith{\iM}{\iC'})}
  % &\to \Oenrich{\E}{\iC}{\Oenrich{\E}{\iC'}{\iM}}\\
  \Oenrich{\E}{\iC}{\Salias{\iP}}
  &\to \Salias{\iP} &\text{when } \Oenrich{\E}{\iC}{\resolve{\E}{\iP}}\to \_\\
  % \Oenrich{\E}{\iC}{\Stransp{\iM}{\iM'}}
  % &\to \Stransp{\Oenrich{\E}{\iC}{\iM}}{\iM'}\\
  \Oenrich{\E}{\iC}{(\Sfunctor{\ix}{\iM}{\iM'})}
  &\to \texttt{fail}\\
  \Oenrich{\E}{\iC}{\Ssig{\ix}{\iS}}
  &\to \Ssig{\ix}{\Oenrich{\E}{\iC}{\iS}}\\[4mm]
  %
  \Oenrich{\E}{t = \tau'}{\Stype{t}{\tau};\iS}
  &\to \Stype{t}{\tau'};\iS
  &\text{when } \subtyp{\E}{\tau'}{\tau}\\
  \Oenrich{\E}{\ix : \iM'}{\Smodulety{\ix}{\iM};\iS}
  &\to \Smodulety{\ix}{\iM'};\iS
  &\text{when } \submod{\E}{\iM'}{\iM}{\_}\\
  \Oenrich{\E}{\ix.\ip.t = \tau}{\Smodulety{\ix}{\iM};\iS}
  &\to \Smodulety{\ix}{\Oenrich{\E}{\ip.t = \tau}{\iM}};\iS\\
  \Oenrich{\E}{\ix.\ip : \iM'}{\Smodulety{\ix}{\iM};\iS}
  &\to \Smodulety{\ix}{\Oenrich{\E}{\ip : \iM'}{\iM}};\iS\\
  \Oenrich{\E}{\iC}{(\iD;\iS)}
  &\to \iD;\Oenrich{\E}{\iC}{\iS}
  &\text{when } id(\iD) \text{ is not a prefix of } id(\iC)\\[4mm]
  %
  \Oenrich{\E}{\iC}{(\Sstrengthen{\ip}{\iM})}
  &\to \Oenrich{\E}{\iC}{\Ostrengthen{\E}{\ip}{\iM}}\\
  \Oenrich{\E}{\iC}{\Senrich{\iC'}{\iM}}
  &\to \Oenrich{\E}{\iC}{\Oenrich{\E}{\iC'}{\iM}}\\
\end{align*}\vspace{-3mm}

%%% Local Variables:
%%% mode: latex
%%% TeX-master: "../main.tex"
%%% End:


  \caption{Enrichment operation -- $\Senrich{\iC}{\iM}$}
  \label{module:enrich}
\end{figure}

\begin{figure}[tbp]
  \begin{align*}
  \Osubst{\E}{\ix : \Salias{\iP}}{\iM}
  &\to \subst{\ix}{\iP}{\iM}\\
  \Osubst{\E}{\ix : \iM}{\iM'}
  &\to \iM' &\texttt{ when } \ix\notin FV(\iM')
\end{align*}
\TODO{Provide something more elaborate}

%%% Local Variables:
%%% mode: latex
%%% TeX-master: "../main.tex"
%%% End:

  \caption{Subst operation -- $\Ssubst{\ix : \iM}{\iM'}$}
  \label{module:subst}
\end{figure}

\begin{figure}[tbp]
  \begin{subfigure}[t]{\linewidth}
    \begin{mathpar}
  \inferrule
  { (\Smodulety{\ix}{\iM}) \in \E }
  { \lookup{\E}{\ix} = \iM }
  \and
  \inferrule
  { \Oshape{\E}{\lookup{\E}{\iP}} = \Ssig{\iy}{\iS_1; \Smodulety{\ix}{\iM}; \iS_2} }
  { \lookup{\E}{\iP.\ix} = \subst{\iy}{\iP}{\iM} }
  \\
  \inferrule
  { (\Smoduletype{\iX}{\iM}) \in \E }
  { \lookup{\E}{\iX} = \iM }
  \and
  \inferrule
  { \Oshape{\E}{\lookup{\E}{\iP}} =
    \Ssig{\iy}{\iS_1; \Smoduletype{\iX}{\iM}; \iS_2} }
  { \lookup{\E}{\iP.\iX} = \subst{\iy}{\iP}{\iM} }
  \and
  \inferrule
  { \lookup{\E}{\iP} = \iM \\
    \submod{\E}{\iM}{\iM'}{\iM_r}
  }
  { \lookup{\E}{\Stransp{\iP}{\iM'}} = \iM_r }
  \and
  \inferrule
  { \Oshape{\E}{\lookup{\E}{\iP_f}} = \Sfunctor{\ix}{\iM_a}{\iM_r} \\
    \submod{\E}{\Salias{\iP_a}}{\iM_a}{\iM}
  }
  { \lookup{\E}{\Sapp{\iP_f}{\iP_a}} =
    {\Ssubst{\iy : \iM}{\subst{\ix}{\iy}{\iM_r}}} }
\end{mathpar}
%%% Local Variables:
%%% mode: latex
%%% TeX-master: "../main"
%%% End:
\vspace{-3mm}
    \caption{Lookup rules -- $\lookup{\E}{\ip} = \iM$}
    \label{module:lookup}
  \end{subfigure}
  \begin{subfigure}[t]{0.5\linewidth}
    \begin{align*}
      \Onormalize{\E}{\iP}
      &=
        \begin{dcases}
          \Onormalize{\E}{\iP'} & \text{when } \lookup{\E}{\iP} = \Salias{\iP'} \\
          \iP & \text{otherwise}
        \end{dcases}
      \\[1em]
      \resolve{\E}{\iP}
      &= \Sstrengthen{\Onormalize{\E}{\iP}}{\lookup{\E}{\Onormalize{\E}{\iP}}}
    \end{align*}
    \caption{Path normalization and resolution}
    \label{module:resolution}
  \end{subfigure}
  \begin{subfigure}[t]{0.45\linewidth}
    \begin{align*}
      \Oshape{\E}{\iM} &= \Oshape{\E}{\Oreduce{\E}{\iM}}\\
      \Oshape{\E}{\Salias{\iP}} &= \Oshape{\E}{\resolve{\E}{\iP}}\\
      \Oshape{\E}{\Ssig{\ix}{\iS}} &= \Ssig{\ix}{\iS}\\
      \Oshape{\E}{\Sfunctor{\ix}{\iM}{\iM'}} &= \Sfunctor{\ix}{\iM}{\iM'}
    \end{align*}
    \caption{Shape resolution -- $\Oshape{\E}{\iM} = \iM'$}
    \label{module:shape}
  \end{subfigure}
  \caption{Environment access, normalization and resolution}
  \label{module:envaccess}
% \end{figure}

% % \begin{figure}[p]
%   \begin{mathpar}
  \inferrule
  { \wfm{\Env}{\Sm} }
  { \wfm{\Env}{\signature{\Sm}} }
  \and
  \inferrule
  { }
  { \wfm{\Env}{\emptym} }
  \and
  \inferrule
  { \wfm[\loc]{\Env}{\Mm_a} \\
    x_i \notin \bv[\loc]{\Env} \\
    \wfm[\loc]{\Env;\bindingm[\loc]{X_i}{\Mm_a}}{\Mm_r}
  }
  { \wfm[\loc]{\Env}{\functor{X_i}{\Mm_a}{\Mm_r}} }
  \and
  % Mixed Functors
  \addedrule{
    \inferrule[\added{MixedFunctor}]
    { \wfm[\sideX]{\Env}{\Mm_a} \\
      X_i \notin \bv[\sideX]{\Env} \\
      \wfm[\sideX]{\Env;\bindingm[\sideX]{X_i}{\Mm_a}}{\Mm_r}
    }
    { \wfm[\sideX]{\Env}{\functor[\sideX]{X_i}{\Mm_a}{\Mm_r}} }
    \and
  }
  \inferrule
  { \type{t_i} \notin \bv{\Env} \\
    \wfm[\mloc]{\Env;\bindingabs{\typicalt}}{\Sm}\\
    \added{\canbedefloc{\mloc}{\loc}}
  }
  { \wfm[\mloc]{\Env}{\typeabsm{\typicalt};\ \Sm} }
  \and
  \inferrule
  { \wfm{\Env}{\Mm} \\
    x_i \notin \bv[\mloc]{\Env} \\
    \wfm[\mloc']{\Env;\bindingm[\mloc]{X_i}{\Mm}}{\Sm}\\
    \added{\canbedefloc{\mloc'}{\mloc}}\\
    \added{\forall \mloc'' \in \locs{\Mm}.\ \canuseloc{\mloc''}{\mloc}}
  }
  { \wfm[\mloc']{\Env}{\moduletym{X_i}{\Mm};\ \Sm} }
  \and
  \inferrule
  { \wf{\Env}{\tau} \\
    \type{t_i} \notin \bv{\Env} \\
    \wfm[\mloc]{\Env;\bindingty{\typicalt}{\tau}}{\Sm}\\
    \added{\canbedefloc{\mloc}{\loc}}
  }
  { \wfm[\mloc]{\Env}{\typem{\typicalt}{\tau};\ \Sm} }
  \and
  \inferrule
  { \wf{\Env}{\tau} \\
    x_i \notin \bv{\Env} \\
    \wfm[\mloc]{\Env;\binding{x_i}{\tau}}{\Sm}\\
    \added{\canbedefloc{\mloc}{\loc}}
  }
  { \wfm[\mloc]{\Env}{\valm{x_i}{\tau};\ \Sm} }
\end{mathpar}

%%% Local Variables:
%%% mode: latex
%%% TeX-master: "../main"
%%% End:
\vspace{-3mm}
%   \caption{Module type validity rules -- $\wfm{\Env}{\Mm}$}
%   \label{module:validity}
\end{figure}



%%% Local Variables:
%%% mode: latex
%%% TeX-command-extra-options: "-shell-escape"
%%% TeX-master: "main.tex"
%%% End:


% \begin{theorem}[Separate Typechecking]\label{ml:separation}
%   Given a list of module declarations that form a typed program, there exists
%   an order such that each module can be typechecked with only knowledge
%   of the type of the previous modules.

%   More formally,
%   given a list of $n$ declarations $\id_i$ and a signature $\iS$ such that
%   \[
%     \wtm{}{(\id_1;\dots;\id_n)}{\iS}
%   \]
%   then there exists $n$ definitions $\iD_i$ and a permutation $\pi$ such that
%   \begin{align*}
%     \forall i < n,\ &
%     \wtm{\iD_{1};\dots;\iD_{i}}{\id_{i+1}}{\iD_{i+1}}&
%     \submod{}{\iD_{\pi(1)};\dots;\iD_{\pi(n)}}{\iS}
%   \end{align*}
% \end{theorem}


\end{document}


%%% Local Variables:
%%% mode: latex
%%% TeX-command-extra-options: "-shell-escape"
%%% TeX-master: t
%%% End:
