\documentclass{article}

% Encoding and lang
\usepackage[T1]{fontenc}
\usepackage[utf8]{inputenc}
\usepackage[english]{babel}

% Graphical packages
\usepackage{graphicx}
\usepackage{xcolor}
\usepackage{xspace}

% Math
\usepackage{amsmath}
\usepackage{amsfonts}
\usepackage{amssymb}
\usepackage{amsthm}
\usepackage{thm-restate}
% \usepackage{mathrsfs}
\usepackage{mathtools}
\usepackage{textcomp}
\usepackage{gensymb}
% \usepackage{textgreek}
\usepackage{multicol}
\usepackage{ stmaryrd }

\usepackage{fullpage}

\theoremstyle{definition}
\newtheorem{theorem}{Theorem}
\newtheorem{lemma}{Lemma}
\newtheorem{corollary}{Corollary}
\newtheorem{definition}{Definition}

\usepackage{listings}
\usepackage[scaled=0.85]{DejaVuSansMono}

\usepackage{caption}
\usepackage{subcaption}
\usepackage[inline,shortlabels]{enumitem}
\setlist{leftmargin=*,noitemsep}
\usepackage{array}
\usepackage{bm}%Decent bolding for math symbols

\usepackage{natbib}% Good citations and bibliography
\usepackage{mathpartir} % Syntax trees

\usepackage[pdfusetitle,colorlinks=true,citecolor=Rhodamine]{hyperref}
\usepackage[noabbrev,capitalize,nameinlink]{cleveref}
\SetLabelAlign{parright}{\parbox[t]{\labelwidth}{\raggedleft#1}}

\newcommand\TODO[1]{{\textbf{\color{red}{TODO: #1}}}}

\title{A parsimonious module system}
\author{TODO}
\date{}

\newcommand\htag[1]{\shortintertext{\textbf{#1}}}
\newcommand\ddotop{\operatorname{:}}
\newcommand\Multi[2][{}]{\overline{#2}}
\newcommand\rg[2]{\left[#1;#2\right]}

%% Term
\newcommand\im{M}
\newcommand\iv{V}
\newcommand\is{S}
\newcommand\id{D}
\newcommand\ip{P}
\newcommand\ix{X}
\newcommand\iy{Y}

%% Types
\newcommand\iM{\mathcal M}
\newcommand\iMO{{\mathcal M^\circ}}
\newcommand\iS{\mathcal S}
\newcommand\iD{\mathcal D}
\newcommand\iX{\mathcal X}
\newcommand\iC{\mathcal C}
\newcommand\iP{\mathcal P}

%% Syntax
\newcommand\Sapp[2]{#1(#2)}
\newcommand\Sconstraint[2]{(#1 \ddotop #2)}
\newcommand\Stransp[2]{(#1 \operatorname{<:} #2)}
\newcommand\Sfunctor[3]{\Sconstraint{#1}{#2} \to #3}
\newcommand\Salias[1]{(= #1)}
\newcommand\Swith[2]{#1\ \mathtt{with}\ #2}

\newcommand\Sstruct[2]{\mathtt{struct}_{#1}\ #2\ \mathtt{end}}
\newcommand\Ssig[2]{\mathtt{sig}_{#1}\ #2\ \mathtt{end}}
\newcommand\Sempty\varepsilon

\newcommand\Senrich[2]{{#2}_{\color{black}\left[#1\right]}}
\newcommand\Sstrengthen[3]{#3/#2}
\newcommand\Ssubst[2]{\mathtt{let}\ #1\ \mathtt{in}\ #2}

\newcommand\Slet[2]{\mathtt{let}\ #1 = #2}
\newcommand\Sval[2]{\mathtt{val}\ #1 : #2}
\newcommand\Stype[2]{\mathtt{type}\ #1 = #2}
\newcommand\Stypeabs[1]{\mathtt{type}\ #1}
\newcommand\Smodule[2]{\mathtt{module}\ #1 = #2}
\newcommand\Smodulety[2]{\mathtt{module}\ #1 : #2}
\newcommand\Smoduletype[2]{\mathtt{module\ type}\ #1 = #2}

%% Env
\newcommand\E{\Gamma}
\newcommand\lookup[2]{#1(#2)}
\newcommand\resolve[2]{\operatorname{resolve}_{#1}(#2)}


\newcommand\subst[3]{#3\!\left[#1 \mapsto #2\right]}
\newcommand\substs[4]{#4\!\left[#1 \mapsto #2\ \middle|\ #1\in#3\right]}
\newcommand\bv[1]{\operatorname{BV}(#1)}

%% Jugements

\newcommand\iswt{\operatorname{\vdash}}
\newcommand\iswtm{\operatorname{\vdash}}
\newcommand\wt[4][]{#2 \iswt_{#1} #3 \ddotop #4}
\newcommand\wtp[3]{#1 \iswtm #2 \ddotop #3}
\newcommand\wtm[4][]{#2 \iswtm_{#1} #3 \ddotop #4}

% Operation
\newcommand\opeq[3]{#1 \iswtm #2 = #3}
\newcommand\Oenrich[4]{\opeq{#1}{\Senrich{#2}{#3}}{#4}}
\newcommand\Ostrengthen[4]{\opeq{#1}{\Sstrengthen{#2}{#3}}{#4}}
\newcommand\Osubst[4]{\opeq{#1}{\Ssubst{#2}{#3}}{#4}}

\newcommand\Oforce[2]{\operatorname{force}_{#1}(#2)}
\newcommand\Oshape[2]{\operatorname{shape}_{#1}(#2)}
\newcommand\Onormalize[2]{\operatorname{normalize}_{#1}(#2)}

% Welformedness of types and terms
\newcommand\iswf{\operatorname{\vDash}}
\newcommand\wf[3][]{#2 \iswf_{#1} #3}
\newcommand\wfp[2]{#1 \iswf #2}
\newcommand\wfm[3][]{#2 \iswf_{#1} #3}

% Module inclusion
\newcommand\issub{<:}
\newcommand\subtyp[3]{#1 \iswt #2 \issub #3}
\newcommand\submod[4]{#1 \iswtm #2 \issub #3 \leadsto #4}
\newcommand\subpath[4][]{#2 \iswtm_{#1} #3 \lesssim #4}
\newcommand\eqpath[4][]{#2 \iswtm_{#1} #3 = #4}


%%% Local Variables:
%%% mode: latex
%%% TeX-master: "main.tex"
%%% End:


\begin{document}

\maketitle

\begin{figure}[!hb]
  
\begin{subfigure}[t]{0.45\linewidth}
\begin{align*}
  \htag{Path}
  \ip ::=&\ \ix_i \mid \ip.\ix\\
  \iP ::=&\ \ix_i \mid \iP.\ix \mid \Sapp{\iP_1}{\iP_2} \mid \Stransp{\iP}{\iM}
  \htag{Module Expressions}
  \im ::=&\ \ip\tag{Variables}\\
  |&\ \Sconstraint{\im}{\iM}\tag{Opaque Ascription}\\
  |&\ \Stransp{\im}{\iM}\tag{Transparent Ascription}\\
  |&\ \Sapp{\im_1}{\im_2}\tag{Functor application}\\
  |&\ \Sfunctor{\ix_i}{\iM}{\im}\tag{Functor}\\
  |&\ \Sstruct{\is}\tag{Structure}\\
  \htag{Structures}
  \is ::=&\ \Sempty\ |\ \id;\is\\
  \id ::=&\ \Slet{x_i}{e}\tag{Values}\\
  |&\ \Stype{t}{\tau}\tag{Types}\\
  |&\ \Smodule{\ix_i}{\im}\tag{Modules}\\
  |&\ \Smoduletype{\iX_i}{\iM}\tag{Module types}\\
  \htag{Core language}
  e ::=&\ \ip.x \tag{Qualified variable}\\
  |&\ \dots \tag{Other expressions}\\
  \tau ::=&\ {\iP.t} \tag{Qualified type}\\
  |&\ \dots \tag{Other types}
\end{align*}
\end{subfigure}\hfill
\begin{subfigure}[t]{0.5\linewidth}
\begin{align*}
  \htag{Module types}
  \iMO ::=&\ \iX_i\ |\ \iP.\iX\tag{Variables}\\
  |&\ \Salias{\iP}\tag{Alias}\\
  |&\ \Sfunctor{X_i}{\iM_1}{\iM_2}\tag{Functor}\\
  |&\ \Ssig{\iS}\tag{Signature}\\
  \iM ::=&\ \Sstrengthen{l}{\iP}{\iM}\tag{Strengthening}\\
  |&\ \Ssubst{\iX : \iM}{\iM}\tag{Let}\\
  |&\ \Senrich{\iC}{\iM}\tag{Enrichment}\\
  |&\ \iMO\\
  \htag{Substitutions}
  \iC ::=&\ \ip.t = \tau\\
  |&\ \ip : \iM\\
  \htag{Signatures}
  \iS ::=&\ \Sempty\ |\ \iD;\iS\\
  \iD ::=&\ \Sval{x_i}{\tau}\tag{Values}\\
  |&\ \Stype{t}{\tau}\tag{Types}\\
  |&\ \Stypeabs{t}\tag{Abstract types}\\
  |&\ \Smodulety{\ix_i}{\iM}\tag{Modules}\\
  |&\ \Smoduletype{\iX_i}{\iM}\tag{Module types}\\
  \htag{Environments}
    \E ::=&\ \iS
\end{align*}
\end{subfigure}

%%% Local Variables:
%%% mode: latex
%%% TeX-master: "../main"
%%% End:

  \caption{Module language}
  \label{grammar}
\end{figure}

Judgements:\\
\begin{description}[align=right, leftmargin=3.5cm]
\item[$\wtm{\E}{\im}{\iM}$ :]
  The module $\im$ is of type $\iM$ in $\E$.
  See \cref{module:typing}.
\item[$\submod{\E}{\iM}{\iM'}{\iM_r}$ :]
  The module type $\iM$ is a subtype of $\iM'$ in $\E$.
  See \cref{module:subtyping}.
\item[$\Oreduce{\E}{\iM} = \iMO$ :]
  The module type with operations $\iM$ can be reduced to a module
  type without operations $\iMO$.
  See \cref{module:strengthen,module:enrich}.
  \TODO{Change bracket style ...}
\end{description}

Operations:\\
\begin{tabular}{ll}
  $\Sstrengthen{l}{\iP}{\iM}$
  & Strenghten $\iM$ by $\iP$ while ignoring the list of module paths $l$.\\
  $\Senrich{\iC}{\iM}$
  & Enrich the module $\iM$ with the constraints $\iC$.\\
  $\Ssubst{\iX : \iM}{\iM'}$
  & Substitutes $\iX$ by $\iM$ in $\iM'$.
  % $\Oforce{\E}{\iM}$
  % & Forces all the operations in $\iM$ until obtaining a simple module type $\iMO$.
\end{tabular}\\

Environment accesses:\\
\begin{tabular}{ll}
  $\lookup{\E}{\ip}$
  & Lookup the module or module type $\ip$ in the environment (i.e., the signature)
    $\E$.\\
  $\resolve{\E}{\iMO}$
  & Resolve $\iMO$ until it's not a path (i.e., either an arrow or a signature).\\
  $\Onormalize{\E}{\iP}$
  & Normalizes the path $\iP$ in $\E$
\end{tabular}

\begin{figure}[tbp]
  \mprset{sep=1.5em}
  \begin{mathpar}
  % \inferrule[ModVar]
  % { \Smodulety{\ix}{\iM} \in \E }
  % { \wtm{\E}{\ix}{\Salias{\ix}} }
  % \and
  \inferrule[ModVar]
  { \ip \in \E }
  % { \wtm{\E}{\ip.\ix}{\substs{n}{\ip.n}{\bv{\iS_1}}{\iM}} }
  { \wtm{\E}{\ip}{\Salias{\ip}} }
  \and
  % \inferrule[Strength]
  % { \wtm{\E}{\ip}{\iM} }
  % { \wtm{\E}{\ip}{\Sstrengthen{}{\ip}{\iM}} }
  % \and
  \inferrule
  { \wtm{\E}{\im}{\iM'} \\ \submod{\E}{\iM'}{\iM}{\_} }
  { \wtm{\E}{\im}{\iM} }
  \and
  \inferrule
  { \wfm{\E}{\iM} \\ \wtm{\E}{\im}{\iM} }
  { \wtm{\E}{\Sconstraint{\im}{\iM}}{\iM} }
  \and
  \inferrule
  { \wtm{\E}{\im}{\iM} \\ \submod{\E}{\iM}{\iM}{\iM_r} }
  { \wtm{\E}{\Stransp{\im}{\iM}}{\iM_r} }
  %
  \and
  \inferrule
  { \wtm{\E}{\im_f}{\Sfunctor{\ix}{\iM_a}{\iM_r}} \\
    \wtm{\E}{\im}{\iM} \\
    \submod{\E}{\iM}{\iM_a}{\iM_c}
  }
  { \wtm{\E}{\Sapp{\im_f}{\im}}
    {\Ssubst{\ix : \iM_c}{\iM_r}} }
  \and
  % \inferrule
  % { \wtm{\E}{\im_f}{\Sfunctor{\ix}{\iM}{\iM_r}} \\
  %   \wtm{\E}{\im_a}{\iM_a} \\
  %   \submod{\E}{\iM_a}{\iM}
  % }
  % { \wtm{\E}{\Sapp{\im_f}{\im_a}}{\iM_r} }
  % \and
  \inferrule
  { \wfm{\E}{\iM} \\
    X \notin \bv{\E} \\
    \wtm{\E;\Smodulety{X}{\iM}}{\im}{\iM'}
  }
  { \wtm{\E}{\Sfunctor{X}{\iM}{\im}}{\Sfunctor{X}{\iM}{\iM'}} }
  \and
  \inferrule
  { \wtm[\iy]{\E}{\is}{\iS} }
  { \wtm{\E}{\Sstruct{\ix}{\is}}{\Ssig{\ix}{\iS}} }
  \and
  \inferrule
  { }
  { \wtm[\iy]{\E}{\Sempty}{\Sempty} }
  %
  \\
  \inferrule
  { \wt{\E}{e}{\tau} \\
    x \notin \lookup{\E}{\iy} \\
    \wtm[\iy]{\E;\Sval{\iy.x}{\tau}}{\is}{\iS}
  }
  { \wtm[\iy]{\E}{(\Slet{x}{e}; \is)}{(\Sval{x}{\tau}; \iS)} }
  \and
  \inferrule
  { \wtm{\E}{\im}{\iM} \\
    % \submod{\E}{\Sstrengthen{}{\im}{\iM'}}{\iM} \\
    \ix \notin \lookup{\E}{\iy} \\
    \wtm[\iy]{\E;\Smodulety{\iy.\ix}{\iM}}{\is}{\iS}
  }
  { \wtm[\iy]{\E}{(\Smodule{\ix}{\im}; \is)}{(\Smodulety{\ix}{\iM}; \iS)} }
  \and
  \inferrule
  { \wf{\E}{\tau} \\
    t \notin \lookup{\E}{\iy} \\
    \wtm[\iy]{\E;\Stype{\iy.t}{\tau}}{\is}{\iS}
  }
  { \wtm[\iy]{\E}{(\Stype{t}{\tau}; \is)}{(\Stype{t}{\tau}; \iS)} }
  \and
  \inferrule
  { \wfm{\E}{\iM} \\
    \iX \notin \lookup{\E}{\iy} \\
    \wtm[\iy]{\E;\Smoduletype{\iy.\iX}{\iM}}{\is}{\iS}
  }
  { \wtm[\iy]{\E}{(\Smoduletype{\iX}{\iM}; \is)}{(\Smoduletype{\iX}{\iM}; \iS)} }
  % \and
  % \inferrule
  % { \wtm[\sideX]{\E}{(\struct{\is})}{(\signature{\valm[
  % { \wtp{\E}{(\prog{\is})}{\tau} }
\end{mathpar}

%%% Local Variables:
%%% mode: latex
%%% TeX-master: "../main"
%%% End:
\vspace{-3mm}
  \caption{Module typing rules -- $\wtm{\E}{\im}{\iM}$}
  \label{module:typing}
\end{figure}

\begin{figure}[tbp]
  \begin{mathpar}
  \inferrule
  { \submod{\E}{\Oforce{\E}{\iM}}{\Oforce{\E}{\iM'}} }
  { \submod{\E}{\iM}{\iM'}{} }
  \and
  \inferrule
  { \Onormalize{\E}{\ip} = \Onormalize{\E}{\ip'} }
  { \submod{\E}{\Salias{\ip}}{\Salias{\ip'}}{} }
  \and
  % \inferrule
  % { }
  % { \submod{\E}{\ip.\iX}{\ip.\iX} }
  % \and
  % \inferrule
  % { \submod{\E}{\E(\ip)}{\iMO} }
  % { \submod{\E}{\Salias{\ip}}{\iMO} }
  % \and
  % \inferrule
  % { \submod{\E}{\E(\ip.\iX)}{\iMO} }
  % { \submod{\E}{\ip.\iX}{\iMO} }
  % \and
  % \inferrule
  % { \submod{\E}{\iMO}{\E(\ip.\iX)} }
  % { \submod{\E}{\iMO}{\ip.\iX} }
  % \and
  \inferrule
  { \submod{\E}{\resolve{\E}{\ip}}{\iM}{} }
  { \submod{\E}{\Salias{\ip}}{\iMO}{} }
  \and
  \inferrule
  { \submod{\E}{\lookup{\E}{\iP}}{\iMO}{} }
  { \submod{\E}{\iP}{\iMO}{} }
  \and
  \inferrule
  { \submod{\E}{\iMO}{\lookup{\E}{\iP}}{} }
  { \submod{\E}{\iMO}{\iP}{} }
  \and
  \inferrule
  { \Onormalize{\E}{\ip} = \Onormalize{\E}{\ip'} \\
    \submod{\E}{\resolve{\E}{\ip}}{\iM}{\iM''} \\
    \submod{\E}{\iM''}{\iM'}{\iM_r}
  }
  { \submod{\E}
    {\Salias{\Stransp{\ip}{\iM}}}{\Salias{\Stransp{\ip'}{\iM'}}}
    {\Salias{\Stransp{\ip'}{\iM'}}} }
  \and
  \inferrule
  { \subpath{\E}{\ip}{\ip'} \\
    \submod{\E}{\resolve{\E}{\ip}}{\iM}{\iM''} \\
    \submod{\E}{\iM''}{\resolve{\E}{\ip}}{\iM_r}
  }
  { \submod{\E}{\Salias{\Stransp{\ip}{\iM}}}{\Salias{\ip'}}
    {\Salias{\ip'}} }
  \and
  % \inferrule
  % { \submod{\E}{\Multi\iC}{\Multi\iC'} }
  % { \submod{\E}{\Swith{\iM}{\Multi\iC}}{\Swith{\iM}{\Multi\iC'}} }
  % \and
  \inferrule
  { \submod{\E}{\iM'_a}{\iM_a}{\_} \\
    \submod{\E,\Smodulety{X}{\iM'_a}}{\iM_r}{\iM'_r}{\iM''_r}
  }
  { \submod{\E}{\Sfunctor{X}{\iM_a}{\iM_r}}{\Sfunctor{X}{\iM'_a}{\iM'_r}}
    {\Sfunctor{X}{\iM'_a}{\iM''_r}} }
  \and
  %
  \inferrule
  { \pi:\rg{1}{m} \to \rg{1}{n} \\
    \forall i\in[1;m],\ \submod{\E;\iD_1;\dots;\iD_n}{\iD_{\pi(i)}}{\iD'_i}{}
  }
  { \submod{\E}{\Ssig{\iD_1;\dots;\iD_n}}{\Ssig{\iD'_1;\dots;\iD'_m}}{} }
  \and
  %
  \inferrule
  { \subtyp{\E}{\tau_1}{\tau_2}{} }
  { \submod{\E}{(\Sval{x_i}{\tau_1})}{(\Sval{x_i}{\tau_2})}{} }
  \and
  \inferrule
  { \submod{\E}{\iM_1}{\iM_2}{}  }
  { \submod{\E}{(\Smodulety{X_i}{\iM_1})}{(\Smodulety{X_i}{\iM_2})}{} }
  \and
  \inferrule
  { \subtyp{\E}{\tau_1}{\tau_2} }
  { \submod{\E}
    {(\Stype{t_i}{\tau_1})}
    {(\Stype{t_i}{\tau_2})}
    {}
  }
  \and
  \inferrule
  { }
  { \submod{\E}
    {(\Stypeabs{t_i})}
    {(\Stypeabs{t_i})}
    {}
  }
  \and
  \inferrule
  { \subtyp{\E}{t}{\tau} }
  { \submod{\E}
    {(\Stypeabs{t_i})}
    {(\Stype{t_i}{\tau})}
    {}
  }
  \and
  \inferrule
  { }
  { \submod{\E}
    {(\Stype{t_i}{\tau_1})}
    {(\Stypeabs{t_i})}
    {}
  }
\end{mathpar}

%%% Local Variables:
%%% mode: latex
%%% TeX-master: "../main"
%%% End:
\vspace{-3mm}
  \caption{Module subtyping rules -- $\submod{\E}{\iM}{\iM'}{\iM_r}$}
  \label{module:subtyping}
\end{figure}

\begin{figure}[tbp]
  \begin{align*}
  \Ostrengthen{\E}{\iP}{\iX}
  &\to \Ostrengthen{\E}{\iP}{\lookup{\E}{\iX}}\\
  \Ostrengthen{\E}{\iP}{\iP'.\iX}
  &\to \Ostrengthen{\E}{\iP}{\lookup{\E}{\iP'.\iX}}\\
  \Ostrengthen{\E}{\iP}{\Salias{\iP'}}
  &\to \Salias{\iP'}\\
  % \Ostrengthen{\E}{\iP}{(\Swith{\iM}{\iC})}
  % &\to \Swith{(\Ostrengthen{\E}{l,\operatorname{id}(\iC)}{\iP}{\iM})}{\iC}\\
  % &\to \Ostrengthen{\E}{\iP}{\Oenrich{\iC}{\iM}} \\
  \Ostrengthen{\E}{\iP}{(\Sfunctor{\ix}{\iM}{\iM'})}
  &\to \Sfunctor{\ix}{\iM}{\Sstrengthen{\iP(X)}{\iM'}}\\
  % \Ostrengthen{\E}{\iP}{\Stransp{\iM}{\iM'}}
  % &\to \Stransp{\Ostrengthen{\E}{\iP}{\iM}}{\iM'}\\
  \Ostrengthen{\E}{\iP}{\Ssig{\ix}{\iS}}
  &\to \Ssig{\ix}{\Ostrengthen{\E}{\iP}{\iS}}\\[4mm]
  %
  \Ostrengthen{\E}{\iP}{\Stype{t}{\tau};\iS}
  &\to \Stype{t}{\iP.t};\Ostrengthen{\E}{\iP}{\iS} % &\text{when } t\notin l
  \\
  % \Ostrengthen{\E}{\iP}{\Stype{t}{\tau};\iS}
  % &\to \Stype{t}{\tau};(\Ostrengthen{\E}{\iP}{\iS}) &\text{when } t\in l\\
  \Ostrengthen{\E}{\iP}{\Stypeabs{t};\iS}
  &\to \Stype{t}{\iP.t};\Ostrengthen{\E}{\iP}{\iS} % &\text{when } t\notin l
  \\
  % \Ostrengthen{\E}{\iP}{\Stypeabs{t};\iS}
  % &\to \Stypeabs{t};(\Ostrengthen{\E}{\iP}{\iS}) &\text{when } t\in l\\
  % \Ostrengthen{\E}{\iP}{\Smodulety{\ix}{\iM};\iS}
  % &\to \Smodulety{\ix}{\Ostrengthen{\E}{\operatorname{chop}(l,\ix)}{\iP.\ix}{\iM}};
  %   (\Ostrengthen{\E}{\iP}{\iS}) &\text{when } \ix\notin l\\
  \Ostrengthen{\E}{\iP}{\Smodulety{\ix}{\iM};\iS}
  &\to \Smodulety{\ix}{\Salias{\iP.\ix}};
    \Ostrengthen{\E}{\iP}{\iS} % &\text{when } \ix\notin l
  \\
  % \Ostrengthen{\E}{\iP}{\Smodulety{\ix}{\iM};\iS}
  % &\to \Smodulety{\ix}{\iM};
  %   (\Ostrengthen{\E}{\iP}{\iS})  &\text{when } \ix\in l\\
  \Ostrengthen{\E}{\iP}{\Smoduletype{\iX}{\iM};\iS}
  &\to \Smoduletype{\iX}{\iM};
    \Ostrengthen{\E}{\iP}{\iS} \\[4mm]
  %
  \Ostrengthen{\E}{\iP'}{\Sstrengthen{\iP}{\iM}}
  &\to \Ostrengthen{\E}{\iP}{\iM}\\
  \Ostrengthen{\E}{\iP}{\Senrich{\iC}{\iM}}
  &\to \Ostrengthen{\E}{\iP}{\Oenrich{\E}{\iC}{\iM}}\\
\end{align*}\vspace{-3mm}

%%% Local Variables:
%%% mode: latex
%%% TeX-master: "../main.tex"
%%% End:

  \caption{Module strengthening operation -- $\Sstrengthen{l}{\iP}{\iM}$}
  \label{module:strengthen}
\end{figure}

\begin{figure}[tbp]
  \begin{align*}
  \Oenrich{\E}{\iC}{\iX}
  &\to \Oenrich{\E}{\iC}{\lookup{\E}{\iX}}\\
  \Oenrich{\E}{\iC}{\iP.\iX}
  &\to \Oenrich{\E}{\iC}{\lookup{\E}{\iP.\iX}}\\
  % \Oenrich{\E}{\iC}{(\Swith{\iM}{\iC'})}
  % &\to \Oenrich{\E}{\iC}{\Oenrich{\E}{\iC'}{\iM}}\\
  \Oenrich{\E}{\iC}{\Salias{\iP}}
  &\to \Salias{\iP} &\text{when } \Oenrich{\E}{\iC}{\resolve{\E}{\iP}}\to \_\\
  % \Oenrich{\E}{\iC}{\Stransp{\iM}{\iM'}}
  % &\to \Stransp{\Oenrich{\E}{\iC}{\iM}}{\iM'}\\
  \Oenrich{\E}{\iC}{(\Sfunctor{\ix}{\iM}{\iM'})}
  &\to \texttt{fail}\\
  \Oenrich{\E}{\iC}{\Ssig{\ix}{\iS}}
  &\to \Ssig{\ix}{\Oenrich{\E}{\iC}{\iS}}\\[4mm]
  %
  \Oenrich{\E}{t = \tau'}{\Stype{t}{\tau};\iS}
  &\to \Stype{t}{\tau'};\iS
  &\text{when } \subtyp{\E}{\tau'}{\tau}\\
  \Oenrich{\E}{\ix : \iM'}{\Smodulety{\ix}{\iM};\iS}
  &\to \Smodulety{\ix}{\iM'};\iS
  &\text{when } \submod{\E}{\iM'}{\iM}{\_}\\
  \Oenrich{\E}{\ix.\ip.t = \tau}{\Smodulety{\ix}{\iM};\iS}
  &\to \Smodulety{\ix}{\Oenrich{\E}{\ip.t = \tau}{\iM}};\iS\\
  \Oenrich{\E}{\ix.\ip : \iM'}{\Smodulety{\ix}{\iM};\iS}
  &\to \Smodulety{\ix}{\Oenrich{\E}{\ip : \iM'}{\iM}};\iS\\
  \Oenrich{\E}{\iC}{(\iD;\iS)}
  &\to \iD;\Oenrich{\E}{\iC}{\iS}
  &\text{when } id(\iD) \text{ is not a prefix of } id(\iC)\\[4mm]
  %
  \Oenrich{\E}{\iC}{(\Sstrengthen{\ip}{\iM})}
  &\to \Oenrich{\E}{\iC}{\Ostrengthen{\E}{\ip}{\iM}}\\
  \Oenrich{\E}{\iC}{\Senrich{\iC'}{\iM}}
  &\to \Oenrich{\E}{\iC}{\Oenrich{\E}{\iC'}{\iM}}\\
\end{align*}\vspace{-3mm}

%%% Local Variables:
%%% mode: latex
%%% TeX-master: "../main.tex"
%%% End:


  \caption{Enrichment operation -- $\Senrich{\iC}{\iM}$}
  \label{module:enrich}
\end{figure}

\begin{figure}[tbp]
  \begin{mathpar}
  \inferrule
  { \Smodulety{\ix_i}{\iM} \in \E }
  { \lookup{\E}{\ix_i} = \iM }
  \and
  \inferrule[\TODO{Clarify how to obtain signature (normalize+force?)}]
  { \lookup{\E}{\iP} = \Ssig{\iS_1; \Smodulety{\ix}{\iM}; \iS_2} }
  { \lookup{\E}{\iP.\ix} = \substs{n_i}{\iP.n}{\bv{\iS_1}}{\iM} }
  \\
  \inferrule
  { \Smoduletype{\iX_i}{\iM} \in \E }
  { \lookup{\E}{\iX_i} = \iM }
  \and
  \inferrule
  { \lookup{\E}{\iP} = \Ssig{\iS_1; \Smoduletype{\iX}{\iM}; \iS_2} }
  { \lookup{\E}{\iP.\iX} = \substs{n_i}{\iP.n}{\bv{\iS_1}}{\iM} }
  \and
  \inferrule
  { \lookup{\E}{\iP} = \iM \\
    \submod{\E}{\iM}{\iM'}{\iM_r}
  }
  { \lookup{\E}{\Stransp{\iP}{\iM'}} = \iM_r }
  \and
  \inferrule
  { \lookup{\E}{\iP_f} = \Sfunctor{\ix}{\iM_a}{\iM_r} \\
    \submod{\E}{\Salias{\iP_a}}{\iM_a}{\iM}
  }
  { \lookup{\E}{\Sapp{\iP_f}{\iP_a}} =
    \Ssubst{\ix : \iM}{\iM_r} }
\end{mathpar}
%%% Local Variables:
%%% mode: latex
%%% TeX-master: "../main"
%%% End:
\vspace{-3mm}
  \caption{Lookup rules -- $\lookup{\E}{\ip} = \iM$}
  \label{module:lookup}
% \end{figure}

% \begin{figure}[hbt]
  \begin{subfigure}[t]{0.5\linewidth}
    \begin{align*}
      \Oshape{\E}{\iM} &= \Oshape{\E}{\Oreduce{\E}{\iM}}\\
      \Oshape{\E}{\Salias{\iP}} &= \Oshape{\E}{\resolve{\E}{\iP}}\\
      \Oshape{\E}{\Ssig{\ix}{\iS}} &= \Ssig{\ix}{\iS}\\
      \Oshape{\E}{\Sfunctor{\ix}{\iM}{\iM'}} &= \Sfunctor{\ix}{\iM}{\iM'}
    \end{align*}
  \end{subfigure}
  \begin{subfigure}[t]{0.5\linewidth}
    \begin{align*}
      \resolve{\E}{\iP}
      &= \Sstrengthen{}{\Onormalize{\E}{\iP}}{\lookup{\E}{\Onormalize{\E}{\iP}}}
      \\[1em]
      \Onormalize{\E}{\iP}
      &=
        \begin{dcases}
          \Onormalize{\E}{\iP'} & \text{when } \lookup{\E}{\iP} = \Salias{\iP'} \\
          \iP & \text{otherwise}
        \end{dcases}
    \end{align*}
  \end{subfigure}
% \end{figure}

% % \begin{figure}[p]
%   \begin{mathpar}
  \inferrule
  { \wfm{\Env}{\Sm} }
  { \wfm{\Env}{\signature{\Sm}} }
  \and
  \inferrule
  { }
  { \wfm{\Env}{\emptym} }
  \and
  \inferrule
  { \wfm[\loc]{\Env}{\Mm_a} \\
    x_i \notin \bv[\loc]{\Env} \\
    \wfm[\loc]{\Env;\bindingm[\loc]{X_i}{\Mm_a}}{\Mm_r}
  }
  { \wfm[\loc]{\Env}{\functor{X_i}{\Mm_a}{\Mm_r}} }
  \and
  % Mixed Functors
  \addedrule{
    \inferrule[\added{MixedFunctor}]
    { \wfm[\sideX]{\Env}{\Mm_a} \\
      X_i \notin \bv[\sideX]{\Env} \\
      \wfm[\sideX]{\Env;\bindingm[\sideX]{X_i}{\Mm_a}}{\Mm_r}
    }
    { \wfm[\sideX]{\Env}{\functor[\sideX]{X_i}{\Mm_a}{\Mm_r}} }
    \and
  }
  \inferrule
  { \type{t_i} \notin \bv{\Env} \\
    \wfm[\mloc]{\Env;\bindingabs{\typicalt}}{\Sm}\\
    \added{\canbedefloc{\mloc}{\loc}}
  }
  { \wfm[\mloc]{\Env}{\typeabsm{\typicalt};\ \Sm} }
  \and
  \inferrule
  { \wfm{\Env}{\Mm} \\
    x_i \notin \bv[\mloc]{\Env} \\
    \wfm[\mloc']{\Env;\bindingm[\mloc]{X_i}{\Mm}}{\Sm}\\
    \added{\canbedefloc{\mloc'}{\mloc}}\\
    \added{\forall \mloc'' \in \locs{\Mm}.\ \canuseloc{\mloc''}{\mloc}}
  }
  { \wfm[\mloc']{\Env}{\moduletym{X_i}{\Mm};\ \Sm} }
  \and
  \inferrule
  { \wf{\Env}{\tau} \\
    \type{t_i} \notin \bv{\Env} \\
    \wfm[\mloc]{\Env;\bindingty{\typicalt}{\tau}}{\Sm}\\
    \added{\canbedefloc{\mloc}{\loc}}
  }
  { \wfm[\mloc]{\Env}{\typem{\typicalt}{\tau};\ \Sm} }
  \and
  \inferrule
  { \wf{\Env}{\tau} \\
    x_i \notin \bv{\Env} \\
    \wfm[\mloc]{\Env;\binding{x_i}{\tau}}{\Sm}\\
    \added{\canbedefloc{\mloc}{\loc}}
  }
  { \wfm[\mloc]{\Env}{\valm{x_i}{\tau};\ \Sm} }
\end{mathpar}

%%% Local Variables:
%%% mode: latex
%%% TeX-master: "../main"
%%% End:
\vspace{-3mm}
%   \caption{Module type validity rules -- $\wfm{\Env}{\Mm}$}
%   \label{module:validity}
\end{figure}




% \begin{theorem}[Separate Typechecking]\label{ml:separation}
%   Given a list of module declarations that form a typed program, there exists
%   an order such that each module can be typechecked with only knowledge
%   of the type of the previous modules.

%   More formally,
%   given a list of $n$ declarations $\id_i$ and a signature $\iS$ such that
%   \[
%     \wtm{}{(\id_1;\dots;\id_n)}{\iS}
%   \]
%   then there exists $n$ definitions $\iD_i$ and a permutation $\pi$ such that
%   \begin{align*}
%     \forall i < n,\ &
%     \wtm{\iD_{1};\dots;\iD_{i}}{\id_{i+1}}{\iD_{i+1}}&
%     \submod{}{\iD_{\pi(1)};\dots;\iD_{\pi(n)}}{\iS}
%   \end{align*}
% \end{theorem}


\end{document}


%%% Local Variables:
%%% mode: latex
%%% TeX-master: t
%%% End:
