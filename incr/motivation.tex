\section{Motivations}
\label{motivations}

\subsection{The aliasing problem}

Several advanced module systems allow ``module aliases''. If we consider a
declaration of the form:

\begin{minted}{ocaml}
module Alias = Original
\end{minted}

In the usual presentation, {\tt Alias} would have the module type {\tt S},
which is the expanded module type of {\tt Original}.
With module aliases, {\tt Alias} is exactly of type \ocaml{(= Original)},
i.e., a singleton type \cite{XX} that uniquely
identify it as an alias of {\tt Original}. This works well with the applicative
behavior of functor used notably by OCaml,
allowing us to preserve \ocaml{F(Alias) = F(Original)}
for any applicative functor {\tt F}.

Unfortunately, this also leads to subtle issues with functors and subtyping.
Let us consider the following examples:

\begin{minted}{ocaml}
module F (X : S) = struct
  module A = X (* this is not an alias *)
end
\end{minted}

What should be the type of \ocaml/F(M).A/? If we keep the singleton type
\ocaml/A : (= X)/, we would have \ocaml/F(M).A : (= M)/. This would mean that
\ocaml{F(M).A} is exactly the same module as \ocaml{M}. This
is however not the case: indeed, when given as argument to the functor,
\ocaml{M} was restricted, by subtyping, to the signature \ocaml{S}.
Inside the body of the functor, we only get a restricted view of {\tt M} through
the lense of {\tt S} and {\tt A} can only present this restricted view.
In practice, this subtyping is implemented by copy: {\tt X} is a new module
which only contains the fields of {\tt M} that are visible in {\tt S}.
If users outside the functor can observe the equality \ocaml/F(M).A = M/,
they might try to access to fields that were not copied and are thus not
really present in the module.
\TODO{Make all this a bit clearer, probably with a schema}

One solution to this problem is \emph{transparent ascriptions}, a type of ascription
that still hides values, but purposefully ``leaks through'' types and other
static fields.
We propose to add transparent ascriptions directly in module paths, allowing
to use them in module aliases. This gives us an immediate
solution to the identified unsafety while still preserving all the possible sharing.
We obtain the following type for the example above:

\begin{minted}{ocaml}
module F (X : S) : sig
  module A = (X <: S) (* X viewed through S *)
end
\end{minted}

We then have \ocaml/F(M).A = (M <: S)/, which properly exposes the sharing
between {\tt A} and {\tt M}, but restricts the access to dynamic fields that are not
present in {\tt S}.

\subsection{The avoidance problem}

TODO: Explain avoidance problem more

\begin{minted}{ocaml}
module F (X : S) = struct
  type a = X.t list
  type b = X.t * int
end
\end{minted}

What's the type of:
\begin{minted}{ocaml}
module A = F(struct type t = A | B end)
\end{minted}

The typechecker gives us the following un-helpful answer
\begin{minted}{ocaml}
module A : sig type a type b end
\end{minted}

We propose to use local modules:

\begin{minted}{ocaml}
module A : let X : ... in sig
  type a = X.t list
  type b = X.t * int
end
\end{minted}

\subsection{Incrementality in module checking}

Modern practical module systems \cite{Rich calculus, etc} provide
many additional operators not present in the original formulation.
For instance in OCaml, ``with constraints'', noted \ocaml{S with type t = int}
allow to add an additional equality to a module type \ocaml{S}.
\TODO{Cite other operations}.

This operation are so far only syntactic: they are expanded immediately into
a complete signature. For instance 
\begin{minted}{ocaml}
module type S = sig 
  val x : string
  module A : T
end
module M : S with type A.t = int
\end{minted}

is immediately expanded as:
\begin{minted}{ocaml}
module M : sig  val x : string  module A : sig type t = int ... end  end
\end{minted}

This yield a simple semantics, but causes numerous efficiency issues in
practical systems that support large Ecosystems: indeed, the signature
of the expanded version can be much larger than the original one.
Our calculus, on the contrary, only expand when absolutely necessary, and keep
the simplest form otherwise. This gives a presentation
similar to explicit substitutions in more traditional type theories, with
similar problematic and trade-offs: we want to minimize the amount of work,
while still doing enough to ensure we do not delay errors.

The application of this technique throughout our system lead to a novel
presentation of parsimonious subtyping that only pushes ascriptions up
to aliases, but not further, ensuring that we maximize the sharing described
in the previous section.


\subsection{Contributions}

We propose a novel module systems which:
\begin{itemize}
\item Uses \emph{syntactic} rules, making it easier to reason about and
  implement as part of production typechecker with inference and error reporting
\item Support \emph{rich operations} on modules such as strengthening (\cite{XL}),
  with constraints, \dots  
\item Support \emph{transparent ascriptions in path}, which safely and
  conveniently extend
  the applicative behavior of functors to a context with singletons.
\item Computes types and subtyping \emph{parsimoniously},
  using an approach based on explicit substitutions and input/output judgements.
\end{itemize}



%%% Local Variables:
%%% mode: latex
%%% TeX-command-extra-options: "-shell-escape"
%%% TeX-master: "main.tex"
%%% End:
