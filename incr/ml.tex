\section{An ML module calculus}

We now introduce our module calculus.
First, let us defined some conventions:
lowercase meta-variables such as $x$, $e$ or $t$ are used for the core language;
capitalized meta-variables such as $\ix$, $\im$, $\is$ represent modules;
calligraphic meta-variables such as $\iX$, $\iM$, $\iS$ represent module types.

\subsection{Syntax}

The syntax is defined in \cref{grammar}.
The module expression language contains paths $\ip$, which might
be variables $\ix$ or qualified accesses $\ip.\ix$,
parameterized modules i.e.~\emph{functors} ($\Sfunctor{\ix}{\iM}{\im}$),
module type annotations i.e.~\emph{ascriptions}, both
opaque $\Sconstraint{\im}{\iM}$ and transparent $\Stransp{\im}{\iM}$,
and finally structures $\Sstruct{\ix}{\is}$. Structures
contains a list of declarations, noted $\id$, which can be value, type, module or module type bindings. Structures are also annotated with a ``Self'' variable,
which represent the object through which
other field in the module should be accessed.

In our core calculus, module types are stratified. $\iM$ represent module
types with operations, while $\iMO$ are raw module types without initial operations
(i.e., in head normal form). Raw module types can be (qualified)
module type variables $\iX$ or $\iP.\iX$, functor types
$\Sfunctor{\iX}{\iM_1}{\iM_2}$, signatures $\Ssig{\ix}{\iS}$,
and singletons $\Salias{\iP}$.
As with structures, signatures are a list of declarations noted $\iD$, and
a ``Self'' variable.
Allowed operations strengthening a module by a path,
noted $\Sstrengthen{\iP}{\iM}$,
enriching a module with additional constraints, noted $\Senrich{\iC}{\iM}$,
and locally binding a module name, noted $\Ssubst{\ix : \iM}{\iM'}$.

As described before, there are two kinds of paths.
The first, $\ip$, is
mostly used for \emph{dynamic} components, and is composed of a list of
modules of the form $A.B.C$.
The second, $\iP$, is mostly used for \emph{static} components such as types,
and can additionally contain functor applications and transparent ascriptions,
for instance $F(\ix).\iy$ or $\Stransp{\ix}{\iM}.\iy$

Finally, the core language is left largely undefined in our calculus, except for qualified accesses.

$\subst{\ix}{\iy}{\iM}$ denotes the substitution of variable $\ix$ by variable $\iy$ in $\iM$. We only substitute variables by other variables.

\begin{figure}[!hb]
  
\begin{subfigure}[t]{0.45\linewidth}
\begin{align*}
  \htag{Path}
  \ip ::=&\ \ix_i \mid \ip.\ix\\
  \iP ::=&\ \ix_i \mid \iP.\ix \mid \Sapp{\iP_1}{\iP_2} \mid \Stransp{\iP}{\iM}
  \htag{Module Expressions}
  \im ::=&\ \ip\tag{Variables}\\
  |&\ \Sconstraint{\im}{\iM}\tag{Opaque Ascription}\\
  |&\ \Stransp{\im}{\iM}\tag{Transparent Ascription}\\
  |&\ \Sapp{\im_1}{\im_2}\tag{Functor application}\\
  |&\ \Sfunctor{\ix_i}{\iM}{\im}\tag{Functor}\\
  |&\ \Sstruct{\is}\tag{Structure}\\
  \htag{Structures}
  \is ::=&\ \Sempty\ |\ \id;\is\\
  \id ::=&\ \Slet{x_i}{e}\tag{Values}\\
  |&\ \Stype{t}{\tau}\tag{Types}\\
  |&\ \Smodule{\ix_i}{\im}\tag{Modules}\\
  |&\ \Smoduletype{\iX_i}{\iM}\tag{Module types}\\
  \htag{Core language}
  e ::=&\ \ip.x \tag{Qualified variable}\\
  |&\ \dots \tag{Other expressions}\\
  \tau ::=&\ {\iP.t} \tag{Qualified type}\\
  |&\ \dots \tag{Other types}
\end{align*}
\end{subfigure}\hfill
\begin{subfigure}[t]{0.5\linewidth}
\begin{align*}
  \htag{Module types}
  \iMO ::=&\ \iX_i\ |\ \iP.\iX\tag{Variables}\\
  |&\ \Salias{\iP}\tag{Alias}\\
  |&\ \Sfunctor{X_i}{\iM_1}{\iM_2}\tag{Functor}\\
  |&\ \Ssig{\iS}\tag{Signature}\\
  \iM ::=&\ \Sstrengthen{l}{\iP}{\iM}\tag{Strengthening}\\
  |&\ \Ssubst{\iX : \iM}{\iM}\tag{Let}\\
  |&\ \Senrich{\iC}{\iM}\tag{Enrichment}\\
  |&\ \iMO\\
  \htag{Substitutions}
  \iC ::=&\ \ip.t = \tau\\
  |&\ \ip : \iM\\
  \htag{Signatures}
  \iS ::=&\ \Sempty\ |\ \iD;\iS\\
  \iD ::=&\ \Sval{x_i}{\tau}\tag{Values}\\
  |&\ \Stype{t}{\tau}\tag{Types}\\
  |&\ \Stypeabs{t}\tag{Abstract types}\\
  |&\ \Smodulety{\ix_i}{\iM}\tag{Modules}\\
  |&\ \Smoduletype{\iX_i}{\iM}\tag{Module types}\\
  \htag{Environments}
    \E ::=&\ \iS
\end{align*}
\end{subfigure}

%%% Local Variables:
%%% mode: latex
%%% TeX-master: "../main"
%%% End:

  \caption{Module language}
  \label{grammar}
\end{figure}


\subsection{Notations}

Let us now define a few notations that we will use in the rest of this
article. We will then describe how these various construction interact
on specific features.

We define the following type judgements on modules. Note that the subtyping
judgement also \emph{returns} a module type, which is the result
of the transparent ascription between the two modules.
\begin{description}[align=right, leftmargin=3.5cm]
\item[$\wtm{\E}{\im}{\iM}$ :]
  The module $\im$ is of type $\iM$ in $\E$.
  See \cref{module:typing}.
\item[$\submod{\E}{\iM}{\iM'}{\iM_r}$ :]
  The module type $\iM$ is a subtype of $\iM'$ in environment $\E$,
  and their ascription is $\iM_r$.
  See \cref{module:subtyping}.
\item[$\Oreduce{\E}{\iM} = \iMO$ :]
  The module type with operations $\iM$ can be reduced to a module
  type without operations $\iMO$ in environment $\E$.
  See \cref{module:strengthen,module:enrich,module:subst}.
  \TODO{Change bracket style ...}
\end{description}

In our calculus, environments are simply signature on which we can
``query'' fields.
To properly model our module calculus, we need to carefully consider
how these environments are accessed. For this purpose, we consider the following
operations on environments, which are defined in \cref{module:envaccess}
\begin{description}[align=right, leftmargin=3.5cm]
\item[$\lookup{\E}{\ip}$ :]
  Lookup the path $\ip$ in $\E$ to a module or module type.
\item[$\Onormalize{\E}{\iP}$ :]
  Normalizes the path $\iP$ in $\E$ to a canonical path.
\item[$\resolve{\E}{\iP}$ :]
  Resolve the path $\iP$ in $\E$ to its shape (i.e., either an arrow or a signature).
\item[$\Oshape{\E}{\iM}$ :]
  Resolve the module $\iM$ in $\E$ to its shape (i.e., either an arrow or a signature).
\end{description}

\subsection{Structures and qualified accesses}

\subsection{Subtyping and transparent ascriptions}

\subsection{Functors}

\subsection{Module operations}

\begin{figure}[tbp]
  \mprset{sep=1.5em}
  \begin{mathpar}
  % \inferrule[ModVar]
  % { \Smodulety{\ix}{\iM} \in \E }
  % { \wtm{\E}{\ix}{\Salias{\ix}} }
  % \and
  \inferrule[ModVar]
  { \ip \in \E }
  % { \wtm{\E}{\ip.\ix}{\substs{n}{\ip.n}{\bv{\iS_1}}{\iM}} }
  { \wtm{\E}{\ip}{\Salias{\ip}} }
  \and
  % \inferrule[Strength]
  % { \wtm{\E}{\ip}{\iM} }
  % { \wtm{\E}{\ip}{\Sstrengthen{}{\ip}{\iM}} }
  % \and
  \inferrule
  { \wtm{\E}{\im}{\iM'} \\ \submod{\E}{\iM'}{\iM}{\_} }
  { \wtm{\E}{\im}{\iM} }
  \and
  \inferrule
  { \wfm{\E}{\iM} \\ \wtm{\E}{\im}{\iM} }
  { \wtm{\E}{\Sconstraint{\im}{\iM}}{\iM} }
  \and
  \inferrule
  { \wtm{\E}{\im}{\iM} \\ \submod{\E}{\iM}{\iM}{\iM_r} }
  { \wtm{\E}{\Stransp{\im}{\iM}}{\iM_r} }
  %
  \and
  \inferrule
  { \wtm{\E}{\im_f}{\Sfunctor{\ix}{\iM_a}{\iM_r}} \\
    \wtm{\E}{\im}{\iM} \\
    \submod{\E}{\iM}{\iM_a}{\iM_c}
  }
  { \wtm{\E}{\Sapp{\im_f}{\im}}
    {\Ssubst{\ix : \iM_c}{\iM_r}} }
  \and
  % \inferrule
  % { \wtm{\E}{\im_f}{\Sfunctor{\ix}{\iM}{\iM_r}} \\
  %   \wtm{\E}{\im_a}{\iM_a} \\
  %   \submod{\E}{\iM_a}{\iM}
  % }
  % { \wtm{\E}{\Sapp{\im_f}{\im_a}}{\iM_r} }
  % \and
  \inferrule
  { \wfm{\E}{\iM} \\
    X \notin \bv{\E} \\
    \wtm{\E;\Smodulety{X}{\iM}}{\im}{\iM'}
  }
  { \wtm{\E}{\Sfunctor{X}{\iM}{\im}}{\Sfunctor{X}{\iM}{\iM'}} }
  \and
  \inferrule
  { \wtm[\iy]{\E}{\is}{\iS} }
  { \wtm{\E}{\Sstruct{\ix}{\is}}{\Ssig{\ix}{\iS}} }
  \and
  \inferrule
  { }
  { \wtm[\iy]{\E}{\Sempty}{\Sempty} }
  %
  \\
  \inferrule
  { \wt{\E}{e}{\tau} \\
    x \notin \lookup{\E}{\iy} \\
    \wtm[\iy]{\E;\Sval{\iy.x}{\tau}}{\is}{\iS}
  }
  { \wtm[\iy]{\E}{(\Slet{x}{e}; \is)}{(\Sval{x}{\tau}; \iS)} }
  \and
  \inferrule
  { \wtm{\E}{\im}{\iM} \\
    % \submod{\E}{\Sstrengthen{}{\im}{\iM'}}{\iM} \\
    \ix \notin \lookup{\E}{\iy} \\
    \wtm[\iy]{\E;\Smodulety{\iy.\ix}{\iM}}{\is}{\iS}
  }
  { \wtm[\iy]{\E}{(\Smodule{\ix}{\im}; \is)}{(\Smodulety{\ix}{\iM}; \iS)} }
  \and
  \inferrule
  { \wf{\E}{\tau} \\
    t \notin \lookup{\E}{\iy} \\
    \wtm[\iy]{\E;\Stype{\iy.t}{\tau}}{\is}{\iS}
  }
  { \wtm[\iy]{\E}{(\Stype{t}{\tau}; \is)}{(\Stype{t}{\tau}; \iS)} }
  \and
  \inferrule
  { \wfm{\E}{\iM} \\
    \iX \notin \lookup{\E}{\iy} \\
    \wtm[\iy]{\E;\Smoduletype{\iy.\iX}{\iM}}{\is}{\iS}
  }
  { \wtm[\iy]{\E}{(\Smoduletype{\iX}{\iM}; \is)}{(\Smoduletype{\iX}{\iM}; \iS)} }
  % \and
  % \inferrule
  % { \wtm[\sideX]{\E}{(\struct{\is})}{(\signature{\valm[
  % { \wtp{\E}{(\prog{\is})}{\tau} }
\end{mathpar}

%%% Local Variables:
%%% mode: latex
%%% TeX-master: "../main"
%%% End:
\vspace{-3mm}
  \caption{Module typing rules -- $\wtm{\E}{\im}{\iM}$}
  \label{module:typing}
\end{figure}

\begin{figure}[tbp]
  \begin{mathpar}
  \inferrule
  { \submod{\E}{\Oforce{\E}{\iM}}{\Oforce{\E}{\iM'}} }
  { \submod{\E}{\iM}{\iM'}{} }
  \and
  \inferrule
  { \Onormalize{\E}{\ip} = \Onormalize{\E}{\ip'} }
  { \submod{\E}{\Salias{\ip}}{\Salias{\ip'}}{} }
  \and
  % \inferrule
  % { }
  % { \submod{\E}{\ip.\iX}{\ip.\iX} }
  % \and
  % \inferrule
  % { \submod{\E}{\E(\ip)}{\iMO} }
  % { \submod{\E}{\Salias{\ip}}{\iMO} }
  % \and
  % \inferrule
  % { \submod{\E}{\E(\ip.\iX)}{\iMO} }
  % { \submod{\E}{\ip.\iX}{\iMO} }
  % \and
  % \inferrule
  % { \submod{\E}{\iMO}{\E(\ip.\iX)} }
  % { \submod{\E}{\iMO}{\ip.\iX} }
  % \and
  \inferrule
  { \submod{\E}{\resolve{\E}{\ip}}{\iM}{} }
  { \submod{\E}{\Salias{\ip}}{\iMO}{} }
  \and
  \inferrule
  { \submod{\E}{\lookup{\E}{\iP}}{\iMO}{} }
  { \submod{\E}{\iP}{\iMO}{} }
  \and
  \inferrule
  { \submod{\E}{\iMO}{\lookup{\E}{\iP}}{} }
  { \submod{\E}{\iMO}{\iP}{} }
  \and
  \inferrule
  { \Onormalize{\E}{\ip} = \Onormalize{\E}{\ip'} \\
    \submod{\E}{\resolve{\E}{\ip}}{\iM}{\iM''} \\
    \submod{\E}{\iM''}{\iM'}{\iM_r}
  }
  { \submod{\E}
    {\Salias{\Stransp{\ip}{\iM}}}{\Salias{\Stransp{\ip'}{\iM'}}}
    {\Salias{\Stransp{\ip'}{\iM'}}} }
  \and
  \inferrule
  { \subpath{\E}{\ip}{\ip'} \\
    \submod{\E}{\resolve{\E}{\ip}}{\iM}{\iM''} \\
    \submod{\E}{\iM''}{\resolve{\E}{\ip}}{\iM_r}
  }
  { \submod{\E}{\Salias{\Stransp{\ip}{\iM}}}{\Salias{\ip'}}
    {\Salias{\ip'}} }
  \and
  % \inferrule
  % { \submod{\E}{\Multi\iC}{\Multi\iC'} }
  % { \submod{\E}{\Swith{\iM}{\Multi\iC}}{\Swith{\iM}{\Multi\iC'}} }
  % \and
  \inferrule
  { \submod{\E}{\iM'_a}{\iM_a}{\_} \\
    \submod{\E,\Smodulety{X}{\iM'_a}}{\iM_r}{\iM'_r}{\iM''_r}
  }
  { \submod{\E}{\Sfunctor{X}{\iM_a}{\iM_r}}{\Sfunctor{X}{\iM'_a}{\iM'_r}}
    {\Sfunctor{X}{\iM'_a}{\iM''_r}} }
  \and
  %
  \inferrule
  { \pi:\rg{1}{m} \to \rg{1}{n} \\
    \forall i\in[1;m],\ \submod{\E;\iD_1;\dots;\iD_n}{\iD_{\pi(i)}}{\iD'_i}{}
  }
  { \submod{\E}{\Ssig{\iD_1;\dots;\iD_n}}{\Ssig{\iD'_1;\dots;\iD'_m}}{} }
  \and
  %
  \inferrule
  { \subtyp{\E}{\tau_1}{\tau_2}{} }
  { \submod{\E}{(\Sval{x_i}{\tau_1})}{(\Sval{x_i}{\tau_2})}{} }
  \and
  \inferrule
  { \submod{\E}{\iM_1}{\iM_2}{}  }
  { \submod{\E}{(\Smodulety{X_i}{\iM_1})}{(\Smodulety{X_i}{\iM_2})}{} }
  \and
  \inferrule
  { \subtyp{\E}{\tau_1}{\tau_2} }
  { \submod{\E}
    {(\Stype{t_i}{\tau_1})}
    {(\Stype{t_i}{\tau_2})}
    {}
  }
  \and
  \inferrule
  { }
  { \submod{\E}
    {(\Stypeabs{t_i})}
    {(\Stypeabs{t_i})}
    {}
  }
  \and
  \inferrule
  { \subtyp{\E}{t}{\tau} }
  { \submod{\E}
    {(\Stypeabs{t_i})}
    {(\Stype{t_i}{\tau})}
    {}
  }
  \and
  \inferrule
  { }
  { \submod{\E}
    {(\Stype{t_i}{\tau_1})}
    {(\Stypeabs{t_i})}
    {}
  }
\end{mathpar}

%%% Local Variables:
%%% mode: latex
%%% TeX-master: "../main"
%%% End:
\vspace{-3mm}
  \caption{Module subtyping rules -- $\submod{\E}{\iM}{\iM'}{\iM_r}$}
  \label{module:subtyping}
\end{figure}

\begin{figure}[tbp]
  \begin{align*}
  \Ostrengthen{\E}{\iP}{\iX}
  &\to \Ostrengthen{\E}{\iP}{\lookup{\E}{\iX}}\\
  \Ostrengthen{\E}{\iP}{\iP'.\iX}
  &\to \Ostrengthen{\E}{\iP}{\lookup{\E}{\iP'.\iX}}\\
  \Ostrengthen{\E}{\iP}{\Salias{\iP'}}
  &\to \Salias{\iP'}\\
  % \Ostrengthen{\E}{\iP}{(\Swith{\iM}{\iC})}
  % &\to \Swith{(\Ostrengthen{\E}{l,\operatorname{id}(\iC)}{\iP}{\iM})}{\iC}\\
  % &\to \Ostrengthen{\E}{\iP}{\Oenrich{\iC}{\iM}} \\
  \Ostrengthen{\E}{\iP}{(\Sfunctor{\ix}{\iM}{\iM'})}
  &\to \Sfunctor{\ix}{\iM}{\Sstrengthen{\iP(X)}{\iM'}}\\
  % \Ostrengthen{\E}{\iP}{\Stransp{\iM}{\iM'}}
  % &\to \Stransp{\Ostrengthen{\E}{\iP}{\iM}}{\iM'}\\
  \Ostrengthen{\E}{\iP}{\Ssig{\ix}{\iS}}
  &\to \Ssig{\ix}{\Ostrengthen{\E}{\iP}{\iS}}\\[4mm]
  %
  \Ostrengthen{\E}{\iP}{\Stype{t}{\tau};\iS}
  &\to \Stype{t}{\iP.t};\Ostrengthen{\E}{\iP}{\iS} % &\text{when } t\notin l
  \\
  % \Ostrengthen{\E}{\iP}{\Stype{t}{\tau};\iS}
  % &\to \Stype{t}{\tau};(\Ostrengthen{\E}{\iP}{\iS}) &\text{when } t\in l\\
  \Ostrengthen{\E}{\iP}{\Stypeabs{t};\iS}
  &\to \Stype{t}{\iP.t};\Ostrengthen{\E}{\iP}{\iS} % &\text{when } t\notin l
  \\
  % \Ostrengthen{\E}{\iP}{\Stypeabs{t};\iS}
  % &\to \Stypeabs{t};(\Ostrengthen{\E}{\iP}{\iS}) &\text{when } t\in l\\
  % \Ostrengthen{\E}{\iP}{\Smodulety{\ix}{\iM};\iS}
  % &\to \Smodulety{\ix}{\Ostrengthen{\E}{\operatorname{chop}(l,\ix)}{\iP.\ix}{\iM}};
  %   (\Ostrengthen{\E}{\iP}{\iS}) &\text{when } \ix\notin l\\
  \Ostrengthen{\E}{\iP}{\Smodulety{\ix}{\iM};\iS}
  &\to \Smodulety{\ix}{\Salias{\iP.\ix}};
    \Ostrengthen{\E}{\iP}{\iS} % &\text{when } \ix\notin l
  \\
  % \Ostrengthen{\E}{\iP}{\Smodulety{\ix}{\iM};\iS}
  % &\to \Smodulety{\ix}{\iM};
  %   (\Ostrengthen{\E}{\iP}{\iS})  &\text{when } \ix\in l\\
  \Ostrengthen{\E}{\iP}{\Smoduletype{\iX}{\iM};\iS}
  &\to \Smoduletype{\iX}{\iM};
    \Ostrengthen{\E}{\iP}{\iS} \\[4mm]
  %
  \Ostrengthen{\E}{\iP'}{\Sstrengthen{\iP}{\iM}}
  &\to \Ostrengthen{\E}{\iP}{\iM}\\
  \Ostrengthen{\E}{\iP}{\Senrich{\iC}{\iM}}
  &\to \Ostrengthen{\E}{\iP}{\Oenrich{\E}{\iC}{\iM}}\\
\end{align*}\vspace{-3mm}

%%% Local Variables:
%%% mode: latex
%%% TeX-master: "../main.tex"
%%% End:

  \caption{Module strengthening operation -- $\Sstrengthen{\iP}{\iM}$}
  \label{module:strengthen}
\end{figure}

\begin{figure}[tbp]
  \begin{align*}
  \Oenrich{\E}{\iC}{\iX}
  &\to \Oenrich{\E}{\iC}{\lookup{\E}{\iX}}\\
  \Oenrich{\E}{\iC}{\iP.\iX}
  &\to \Oenrich{\E}{\iC}{\lookup{\E}{\iP.\iX}}\\
  % \Oenrich{\E}{\iC}{(\Swith{\iM}{\iC'})}
  % &\to \Oenrich{\E}{\iC}{\Oenrich{\E}{\iC'}{\iM}}\\
  \Oenrich{\E}{\iC}{\Salias{\iP}}
  &\to \Salias{\iP} &\text{when } \Oenrich{\E}{\iC}{\resolve{\E}{\iP}}\to \_\\
  % \Oenrich{\E}{\iC}{\Stransp{\iM}{\iM'}}
  % &\to \Stransp{\Oenrich{\E}{\iC}{\iM}}{\iM'}\\
  \Oenrich{\E}{\iC}{(\Sfunctor{\ix}{\iM}{\iM'})}
  &\to \texttt{fail}\\
  \Oenrich{\E}{\iC}{\Ssig{\ix}{\iS}}
  &\to \Ssig{\ix}{\Oenrich{\E}{\iC}{\iS}}\\[4mm]
  %
  \Oenrich{\E}{t = \tau'}{\Stype{t}{\tau};\iS}
  &\to \Stype{t}{\tau'};\iS
  &\text{when } \subtyp{\E}{\tau'}{\tau}\\
  \Oenrich{\E}{\ix : \iM'}{\Smodulety{\ix}{\iM};\iS}
  &\to \Smodulety{\ix}{\iM'};\iS
  &\text{when } \submod{\E}{\iM'}{\iM}{\_}\\
  \Oenrich{\E}{\ix.\ip.t = \tau}{\Smodulety{\ix}{\iM};\iS}
  &\to \Smodulety{\ix}{\Oenrich{\E}{\ip.t = \tau}{\iM}};\iS\\
  \Oenrich{\E}{\ix.\ip : \iM'}{\Smodulety{\ix}{\iM};\iS}
  &\to \Smodulety{\ix}{\Oenrich{\E}{\ip : \iM'}{\iM}};\iS\\
  \Oenrich{\E}{\iC}{(\iD;\iS)}
  &\to \iD;\Oenrich{\E}{\iC}{\iS}
  &\text{when } id(\iD) \text{ is not a prefix of } id(\iC)\\[4mm]
  %
  \Oenrich{\E}{\iC}{(\Sstrengthen{\ip}{\iM})}
  &\to \Oenrich{\E}{\iC}{\Ostrengthen{\E}{\ip}{\iM}}\\
  \Oenrich{\E}{\iC}{\Senrich{\iC'}{\iM}}
  &\to \Oenrich{\E}{\iC}{\Oenrich{\E}{\iC'}{\iM}}\\
\end{align*}\vspace{-3mm}

%%% Local Variables:
%%% mode: latex
%%% TeX-master: "../main.tex"
%%% End:


  \caption{Enrichment operation -- $\Senrich{\iC}{\iM}$}
  \label{module:enrich}
\end{figure}

\begin{figure}[tbp]
  \begin{align*}
  \Osubst{\E}{\ix : \Salias{\iP}}{\iM}
  &\to \subst{\ix}{\iP}{\iM}\\
  \Osubst{\E}{\ix : \iM}{\iM'}
  &\to \iM' &\texttt{ when } \ix\notin FV(\iM')
\end{align*}
\TODO{Provide something more elaborate}

%%% Local Variables:
%%% mode: latex
%%% TeX-master: "../main.tex"
%%% End:

  \caption{Subst operation -- $\Ssubst{\ix : \iM}{\iM'}$}
  \label{module:subst}
\end{figure}

\begin{figure}[tbp]
  \begin{subfigure}[t]{\linewidth}
    \begin{mathpar}
  \inferrule
  { \Smodulety{\ix_i}{\iM} \in \E }
  { \lookup{\E}{\ix_i} = \iM }
  \and
  \inferrule[\TODO{Clarify how to obtain signature (normalize+force?)}]
  { \lookup{\E}{\iP} = \Ssig{\iS_1; \Smodulety{\ix}{\iM}; \iS_2} }
  { \lookup{\E}{\iP.\ix} = \substs{n_i}{\iP.n}{\bv{\iS_1}}{\iM} }
  \\
  \inferrule
  { \Smoduletype{\iX_i}{\iM} \in \E }
  { \lookup{\E}{\iX_i} = \iM }
  \and
  \inferrule
  { \lookup{\E}{\iP} = \Ssig{\iS_1; \Smoduletype{\iX}{\iM}; \iS_2} }
  { \lookup{\E}{\iP.\iX} = \substs{n_i}{\iP.n}{\bv{\iS_1}}{\iM} }
  \and
  \inferrule
  { \lookup{\E}{\iP} = \iM \\
    \submod{\E}{\iM}{\iM'}{\iM_r}
  }
  { \lookup{\E}{\Stransp{\iP}{\iM'}} = \iM_r }
  \and
  \inferrule
  { \lookup{\E}{\iP_f} = \Sfunctor{\ix}{\iM_a}{\iM_r} \\
    \submod{\E}{\Salias{\iP_a}}{\iM_a}{\iM}
  }
  { \lookup{\E}{\Sapp{\iP_f}{\iP_a}} =
    \Ssubst{\ix : \iM}{\iM_r} }
\end{mathpar}
%%% Local Variables:
%%% mode: latex
%%% TeX-master: "../main"
%%% End:
\vspace{-3mm}
    \caption{Lookup rules -- $\lookup{\E}{\ip} = \iM$}
    \label{module:lookup}
  \end{subfigure}
  \begin{subfigure}[t]{0.5\linewidth}
    \begin{align*}
      \Onormalize{\E}{\iP}
      &=
        \begin{dcases}
          \Onormalize{\E}{\iP'} & \text{when } \lookup{\E}{\iP} = \Salias{\iP'} \\
          \iP & \text{otherwise}
        \end{dcases}
      \\[1em]
      \resolve{\E}{\iP}
      &= \Sstrengthen{\Onormalize{\E}{\iP}}{\lookup{\E}{\Onormalize{\E}{\iP}}}
    \end{align*}
    \caption{Path normalization and resolution}
    \label{module:resolution}
  \end{subfigure}
  \begin{subfigure}[t]{0.45\linewidth}
    \begin{align*}
      \Oshape{\E}{\iM} &= \Oshape{\E}{\Oreduce{\E}{\iM}}\\
      \Oshape{\E}{\Salias{\iP}} &= \Oshape{\E}{\resolve{\E}{\iP}}\\
      \Oshape{\E}{\Ssig{\ix}{\iS}} &= \Ssig{\ix}{\iS}\\
      \Oshape{\E}{\Sfunctor{\ix}{\iM}{\iM'}} &= \Sfunctor{\ix}{\iM}{\iM'}
    \end{align*}
    \caption{Shape resolution -- $\Oshape{\E}{\iM} = \iM'$}
    \label{module:shape}
  \end{subfigure}
  \caption{Environment access, normalization and resolution}
  \label{module:envaccess}
% \end{figure}

% % \begin{figure}[p]
%   \begin{mathpar}
  \inferrule
  { \wfm{\Env}{\Sm} }
  { \wfm{\Env}{\signature{\Sm}} }
  \and
  \inferrule
  { }
  { \wfm{\Env}{\emptym} }
  \and
  \inferrule
  { \wfm[\loc]{\Env}{\Mm_a} \\
    x_i \notin \bv[\loc]{\Env} \\
    \wfm[\loc]{\Env;\bindingm[\loc]{X_i}{\Mm_a}}{\Mm_r}
  }
  { \wfm[\loc]{\Env}{\functor{X_i}{\Mm_a}{\Mm_r}} }
  \and
  % Mixed Functors
  \addedrule{
    \inferrule[\added{MixedFunctor}]
    { \wfm[\sideX]{\Env}{\Mm_a} \\
      X_i \notin \bv[\sideX]{\Env} \\
      \wfm[\sideX]{\Env;\bindingm[\sideX]{X_i}{\Mm_a}}{\Mm_r}
    }
    { \wfm[\sideX]{\Env}{\functor[\sideX]{X_i}{\Mm_a}{\Mm_r}} }
    \and
  }
  \inferrule
  { \type{t_i} \notin \bv{\Env} \\
    \wfm[\mloc]{\Env;\bindingabs{\typicalt}}{\Sm}\\
    \added{\canbedefloc{\mloc}{\loc}}
  }
  { \wfm[\mloc]{\Env}{\typeabsm{\typicalt};\ \Sm} }
  \and
  \inferrule
  { \wfm{\Env}{\Mm} \\
    x_i \notin \bv[\mloc]{\Env} \\
    \wfm[\mloc']{\Env;\bindingm[\mloc]{X_i}{\Mm}}{\Sm}\\
    \added{\canbedefloc{\mloc'}{\mloc}}\\
    \added{\forall \mloc'' \in \locs{\Mm}.\ \canuseloc{\mloc''}{\mloc}}
  }
  { \wfm[\mloc']{\Env}{\moduletym{X_i}{\Mm};\ \Sm} }
  \and
  \inferrule
  { \wf{\Env}{\tau} \\
    \type{t_i} \notin \bv{\Env} \\
    \wfm[\mloc]{\Env;\bindingty{\typicalt}{\tau}}{\Sm}\\
    \added{\canbedefloc{\mloc}{\loc}}
  }
  { \wfm[\mloc]{\Env}{\typem{\typicalt}{\tau};\ \Sm} }
  \and
  \inferrule
  { \wf{\Env}{\tau} \\
    x_i \notin \bv{\Env} \\
    \wfm[\mloc]{\Env;\binding{x_i}{\tau}}{\Sm}\\
    \added{\canbedefloc{\mloc}{\loc}}
  }
  { \wfm[\mloc]{\Env}{\valm{x_i}{\tau};\ \Sm} }
\end{mathpar}

%%% Local Variables:
%%% mode: latex
%%% TeX-master: "../main"
%%% End:
\vspace{-3mm}
%   \caption{Module type validity rules -- $\wfm{\Env}{\Mm}$}
%   \label{module:validity}
\end{figure}



%%% Local Variables:
%%% mode: latex
%%% TeX-master: "main.tex"
%%% End:
