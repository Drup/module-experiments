\section{Introduction}

We propose an \emph{expressive}, \emph{syntactic} and \emph{parsimonious} module system.
By expressive, we mean that we support a rich subset of features that have so far
not been formalized together and allow the user for rich manipulation of modules, notably transparent ascriptions, module aliases, enrichment constraints and
applicative functors. Unlike work like \cite{fing}, our rules are
defined \emph{syntactically}, directly on the concrete syntax. This makes the rule easier to grasp, make deriving an inference algorithm almost immediate, and
ensure that error messages and typing information are easier to surface to the user.
Finally, this system is \emph{parsimonious}, in that it does the minimum amount of
expansion required by using ideas similar to explicit substitutions for
module operations. Module types can be quite large, and limiting the expansion
can simplify error messages and improve the speed of the typechecker.

TODO: Explain functor aliasing.

%%% Local Variables:
%%% mode: latex
%%% TeX-master: t
%%% End:
