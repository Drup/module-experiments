\documentclass[aspectratio=169,dvipsnames,svgnames,10pt]{beamer}
\usepackage{minted}
\usepackage{alltt}
\usepackage[utf8]{inputenc}
\usepackage[english]{babel}
\usepackage{xspace}
\usepackage{cancel}
\usepackage{pifont}%% check mark

\usetheme[sectionpage=none]{metropolis}
\setbeamercovered{transparent=20}
\usepackage[scaled=0.8]{DejaVuSansMono} % A decent mono font


\definecolor{green}{rgb}{0, 0.5, 0.1}
\definecolor{magenta}{rgb}{1, 0, 0}
\newcommand{\error}[1]{\textcolor{red}{#1}}
\newcommand{\ok}[1]{\textcolor{green}{#1}}


\newcommand\Y{{\color{Green}{\ding{52}}}\xspace}
\newcommand\N{{\color{Red}{\ding{56}}}\xspace}
\newcommand\M{{\color{Orange}{\textasciitilde}}\xspace}

\begin{document}

\begin{frame}[t,standout]
  \centering
  \Huge Modules -- Low road\\
  \huge Better error messages via diffing\\
  \large Florian \textsc{Angeletti} and Gabriel \textsc{Radanne}
\end{frame}


\begin{frame}[fragile]{Error messages for modules}

  \textbf{Problem}: Modules errors are often verbose and complicated

  \uncover<2->{
  Remark: Modules are often shallow and their structure is ``list-like''!
  }

  \uncover<3->{
    {\bf Idea}: Use the semantic structure of modules for ``diffing''
  }
\end{frame}

\begin{frame}[fragile,t]\frametitle{}

\begin{minted}{ocaml}
module Graph(Vertex:VERTEX)(Edge:EDGE) = struct ... end

module G = Graph(Label)(Vertex)(Edge)
\end{minted}
\begin{onlyenv}<2>
\begin{alltt}
\error{Error}: Signature mismatch:
       Modules do not match:
         sig type t = string end
       is not included in
         VERTEX
       The value `label' is required but not provided
       The value `create' is required but not provided
       The type `label' is required but not provided
       The value `equal' is required but not provided
       The value `hash' is required but not provided
       The value `compare' is required but not provided
\end{alltt}
\end{onlyenv}

\begin{onlyenv}<3>
\begin{alltt}
{\bfseries{}\color{red}{}Error}: The functor application is ill-typed.
       These arguments:
         {\color{red}{}\bfseries{}Label} {\color{green}{}Vertex} {\color{green}{}Edge}
       do not match these parameters:
         functor {\color{red}{}\bfseries{}} {\color{green}{}(Vertex : VERTEX)} {\color{green}{}(Edge : EDGE)} -> ...
  {\color{red}{}\bfseries{}1.} The following extra argument is provided
      Label : sig type t = string end
  {\color{green}{}2.} Module Vertex matches the expected module type VERTEX
  {\color{green}{}3.} Module Edge matches the expected module type EDGE
\end{alltt}
\end{onlyenv}

\end{frame}

\begin{frame}{}
  
  {\Large\bf Finished work:}

  Full implementation for {\bf functor applications} \texttt{F(A)(B)(C)}
  \begin{itemize}
  \item Detects additions/deletions/mismatch (Levenstein distance)
  \item Solid semantic footing (errors are {\it correct} and {\it useful})
  \item Scales very well in practice
  \item Handles the full complexity of OCaml modules (dependent, variadic, \dots)
  \end{itemize}
  
  Implementation in {\bf OCaml \href{https://github.com/ocaml/ocaml/pull/9331}{PR9331}}. Review in progress.
  
  Presented at {\bf ML workshop 2020}\\
  ``High-level error messages for modules through diffing''
\end{frame}


\begin{frame}{}

  {\Large\bf Ongoing/future work:}

  Apply diffing to other structural ``diffable'' types:
  \begin{itemize}
  \item \textbf{Signatures!}\\
    $\Rightarrow$ Boilerplate done, ``just'' need to write the error messages
  \item objects/polymorphic variants\\
    $\Rightarrow$ Easy in theory
  \item Function applications\\
    $\Rightarrow$ promising, but much harder due to unification/inference
  \item Tree diffing for complete module hierarchies\\
    $\Rightarrow$ Probably overkill
  \end{itemize}
\end{frame}

\begin{frame}[t,standout]
  \centering
  \Huge Modules -- High road\\
  \huge Incremental modules\\
  \large Gabriel \textsc{Radanne}, Thomas \textsc{Refis}, Jacques \textsc{Garrigue} and Didier \textsc{Remy}
\end{frame}


\begin{frame}[fragile]{Status of the module system -- The good}
  Core system has excellent theoretical footing
  \begin{itemize}
  \item Applicative functor -- \texttt{Map.Make(String)}
  \item Module ascription -- \texttt{(M : S)}
  \item Manifest -- \texttt{type foo = Foo.t = MyFoo}
  \item Qualified access -- \texttt{A.v}, \texttt{F(X).t}
  \end{itemize}
  \Y All well understood/formalized
\end{frame}

\begin{frame}[fragile]{Status of the module system -- The less good}
  What about ``recent'' features ?
  \begin{itemize}
  \item Generative functor:
    Well understood \Y
  \item \texttt{with} constraints: 
    No real formalization \N
  \item \texttt{module type of}:
    No formalization, Uncertain semantics \N
  \item First class modules:
    Formalized as ``package types'' \Y ...
    but only in SML \N
  \item Module aliases:
    Well studied in theory as singletons \Y ...
    but in very different settings \N
  \item Private fields:
    Only partial formalization
  \item<2>
    Interactions are not always clear
  \end{itemize}
\end{frame}



\begin{frame}{Another look at the module system}

  \textbf{Goal}: Let's have a new look at module formalization:
  \begin{itemize}
  \item Reconcile the problematic pieces
  \item Think about inference
  \item Fix/improve the known problems
  \item Provide a sure footing for further experimentation (modular implicits, \dots)
  \end{itemize}

  \pause
  \textbf{Non-goal}: Design a new language. 1ML is cool, but we can't use it.

  \pause
  Let's look at some of the big problems!
\end{frame}



\begin{frame}[fragile,fragile]{The Ugly -- Part 1\hfill Aliases and functors}
  
  {\bf Problem}: It is not sound to create an alias of a functor parameter:
\begin{minted}{ocaml}
module F (X : S) = struct
  module A = X (* this is not an alias *)
end
\end{minted}
  {\bf Why}: if the alias is kept, \only<2->{{\color{Grey} we could discover that
  \texttt{F(MyModule).A = MyModule}, and thus discover fields that are not in
  \texttt{S}. Since module coercions are implemented by copies, }}that might lead to segfaults\only<1>{?!?}.

    \begin{onlyenv}<3>
  {\bf Currently}: The type is expanded and module equality is lost:
\begin{minted}{ocaml}
module F (X : S) : sig
  module A : S with type t = X.t
end
\end{minted}
\end{onlyenv}
  
\end{frame}

\begin{frame}[fragile]
  \frametitle{Transparent ascriptions}

  {\bf Solution}: Transparent ascriptions!

  Transparent ascriptions hide the values but keep the types.
\begin{minted}{ocaml}
module F (X : S) : sig
  module A = (X <: S) (* X viewed through S *)
end
\end{minted}

  \pause
  \begin{itemize}
  \item A ``filter'' on the dynamic part of the module
  \item Also in path : \texttt{F(Y <:\ S).A}
  \item Generally useful for ``soft constraints'':
\begin{minted}{ocaml}
module MyKey <: Map.OrderedType = ...
\end{minted}
  \item Essential for modular implicits
  \end{itemize}
  
\end{frame}

\begin{frame}[fragile]
  \frametitle{The Ugly -- Part 2\hfill Eager operations}

  \textbf{Problem}: Many operations on modules are expanded eagerly: \texttt{with}, \texttt{include}, \texttt{module type of}, strengthening, \dots
  \begin{itemize}
  \item Lead to big signatures (Eliom has $>5$Mo {\tt .cmi}, Janestreet has build perf issues, \dots)
  \item Loose the original information, problematic for errors and tools (documentation, notably)
  \end{itemize}

  \pause
  \textbf{Currently}, when you type:
\begin{minted}{ocaml}
module type S = sig 
  val x : string
  module A : T
end
module M : S with type A.t = int
\end{minted}

  the typechecker understands:
\begin{minted}{ocaml}
module M : sig  val x : string  module A : sig type t = int ... end  end
\end{minted}
\end{frame}

\begin{frame}[fragile]{Incrementality}
  
  \textbf{Solution}: Make an \emph{incremental} calculus for modules.
  \begin{itemize}
  \item In a nutshell ``memoization for module type operations''.
  \item Similar to explicit substitutions, but on modules
  \item Need to be extra careful around environments!
  \end{itemize}

  When we typecheck ``\texttt{let y = M.x}'', we push only one step:
\begin{minted}{ocaml}
module M : sig
  val x : string
  module A : T with type t = int
end
\end{minted}
\end{frame}

\begin{frame}{Current status}

  {\bf In progress}:
  \begin{itemize}
  \item Core of the specification is done
  \item Prototype ``clean room'' implementation in progress
  \end{itemize}

  {\bf To be done}:
  \begin{itemize}
  \item Progressively add more and more features
  \item Prove good properties
  \end{itemize}
\end{frame}

\appendix

\begin{frame}[fragile,fragile,fragile]
  \frametitle{The Ugly -- Part 3\hfill The avoidance problem}

  \textbf{Problem} Let's write the this C\&C (cool and contrived) functor:
\begin{minted}{ocaml}
module F (X : S) = struct
  type a = X.t list
  type b = X.t * int
end
\end{minted}

  What's the type of:
\begin{minted}{ocaml}
module A = F(struct type t = A | B end)
\end{minted}
  \pause
  The typechecker gives us:
\begin{minted}{ocaml}
module A : sig type a type b end
\end{minted}
  This is not very useful. 
\end{frame}

\begin{frame}[fragile,fragile]{Local bindings and inference}

  We want to have a \emph{principal module type} for this.
  \begin{itemize}
  \item This type is used for inference
  \item It does not have to appear in user signatures
  \end{itemize}
  \pause
  
  {\bf Solution}: Local module types:
\begin{minted}{ocaml}
module A : let X : ... in sig
  type a = X.t list
  type b = X.t * int
end
\end{minted}

  We can then simplify this further down (incrementally, as before).
\end{frame}

\end{document}

%%% Local Variables:
%%% mode: latex
%%% TeX-command-extra-options: "-shell-escape"
%%% TeX-master: t
%%% End:
